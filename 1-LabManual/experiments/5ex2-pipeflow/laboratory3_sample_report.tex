\documentclass[11pt]{article}

    \usepackage[breakable]{tcolorbox}
    \usepackage{parskip} % Stop auto-indenting (to mimic markdown behaviour)
    

    % Basic figure setup, for now with no caption control since it's done
    % automatically by Pandoc (which extracts ![](path) syntax from Markdown).
    \usepackage{graphicx}
    % Keep aspect ratio if custom image width or height is specified
    \setkeys{Gin}{keepaspectratio}
    % Maintain compatibility with old templates. Remove in nbconvert 6.0
    \let\Oldincludegraphics\includegraphics
    % Ensure that by default, figures have no caption (until we provide a
    % proper Figure object with a Caption API and a way to capture that
    % in the conversion process - todo).
    \usepackage{caption}
    \DeclareCaptionFormat{nocaption}{}
    \captionsetup{format=nocaption,aboveskip=0pt,belowskip=0pt}

    \usepackage{float}
    \floatplacement{figure}{H} % forces figures to be placed at the correct location
    \usepackage{xcolor} % Allow colors to be defined
    \usepackage{enumerate} % Needed for markdown enumerations to work
    \usepackage{geometry} % Used to adjust the document margins
    \usepackage{amsmath} % Equations
    \usepackage{amssymb} % Equations
    \usepackage{textcomp} % defines textquotesingle
    % Hack from http://tex.stackexchange.com/a/47451/13684:
    \AtBeginDocument{%
        \def\PYZsq{\textquotesingle}% Upright quotes in Pygmentized code
    }
    \usepackage{upquote} % Upright quotes for verbatim code
    \usepackage{eurosym} % defines \euro

    \usepackage{iftex}
    \ifPDFTeX
        \usepackage[T1]{fontenc}
        \IfFileExists{alphabeta.sty}{
              \usepackage{alphabeta}
          }{
              \usepackage[mathletters]{ucs}
              \usepackage[utf8x]{inputenc}
          }
    \else
        \usepackage{fontspec}
        \usepackage{unicode-math}
    \fi

    \usepackage{fancyvrb} % verbatim replacement that allows latex
    \usepackage{grffile} % extends the file name processing of package graphics
                         % to support a larger range
    \makeatletter % fix for old versions of grffile with XeLaTeX
    \@ifpackagelater{grffile}{2019/11/01}
    {
      % Do nothing on new versions
    }
    {
      \def\Gread@@xetex#1{%
        \IfFileExists{"\Gin@base".bb}%
        {\Gread@eps{\Gin@base.bb}}%
        {\Gread@@xetex@aux#1}%
      }
    }
    \makeatother
    \usepackage[Export]{adjustbox} % Used to constrain images to a maximum size
    \adjustboxset{max size={0.9\linewidth}{0.9\paperheight}}

    % The hyperref package gives us a pdf with properly built
    % internal navigation ('pdf bookmarks' for the table of contents,
    % internal cross-reference links, web links for URLs, etc.)
    \usepackage{hyperref}
    % The default LaTeX title has an obnoxious amount of whitespace. By default,
    % titling removes some of it. It also provides customization options.
    \usepackage{titling}
    \usepackage{longtable} % longtable support required by pandoc >1.10
    \usepackage{booktabs}  % table support for pandoc > 1.12.2
    \usepackage{array}     % table support for pandoc >= 2.11.3
    \usepackage{calc}      % table minipage width calculation for pandoc >= 2.11.1
    \usepackage[inline]{enumitem} % IRkernel/repr support (it uses the enumerate* environment)
    \usepackage[normalem]{ulem} % ulem is needed to support strikethroughs (\sout)
                                % normalem makes italics be italics, not underlines
    \usepackage{soul}      % strikethrough (\st) support for pandoc >= 3.0.0
    \usepackage{mathrsfs}
    

    
    % Colors for the hyperref package
    \definecolor{urlcolor}{rgb}{0,.145,.698}
    \definecolor{linkcolor}{rgb}{.71,0.21,0.01}
    \definecolor{citecolor}{rgb}{.12,.54,.11}

    % ANSI colors
    \definecolor{ansi-black}{HTML}{3E424D}
    \definecolor{ansi-black-intense}{HTML}{282C36}
    \definecolor{ansi-red}{HTML}{E75C58}
    \definecolor{ansi-red-intense}{HTML}{B22B31}
    \definecolor{ansi-green}{HTML}{00A250}
    \definecolor{ansi-green-intense}{HTML}{007427}
    \definecolor{ansi-yellow}{HTML}{DDB62B}
    \definecolor{ansi-yellow-intense}{HTML}{B27D12}
    \definecolor{ansi-blue}{HTML}{208FFB}
    \definecolor{ansi-blue-intense}{HTML}{0065CA}
    \definecolor{ansi-magenta}{HTML}{D160C4}
    \definecolor{ansi-magenta-intense}{HTML}{A03196}
    \definecolor{ansi-cyan}{HTML}{60C6C8}
    \definecolor{ansi-cyan-intense}{HTML}{258F8F}
    \definecolor{ansi-white}{HTML}{C5C1B4}
    \definecolor{ansi-white-intense}{HTML}{A1A6B2}
    \definecolor{ansi-default-inverse-fg}{HTML}{FFFFFF}
    \definecolor{ansi-default-inverse-bg}{HTML}{000000}

    % common color for the border for error outputs.
    \definecolor{outerrorbackground}{HTML}{FFDFDF}

    % commands and environments needed by pandoc snippets
    % extracted from the output of `pandoc -s`
    \providecommand{\tightlist}{%
      \setlength{\itemsep}{0pt}\setlength{\parskip}{0pt}}
    \DefineVerbatimEnvironment{Highlighting}{Verbatim}{commandchars=\\\{\}}
    % Add ',fontsize=\small' for more characters per line
    \newenvironment{Shaded}{}{}
    \newcommand{\KeywordTok}[1]{\textcolor[rgb]{0.00,0.44,0.13}{\textbf{{#1}}}}
    \newcommand{\DataTypeTok}[1]{\textcolor[rgb]{0.56,0.13,0.00}{{#1}}}
    \newcommand{\DecValTok}[1]{\textcolor[rgb]{0.25,0.63,0.44}{{#1}}}
    \newcommand{\BaseNTok}[1]{\textcolor[rgb]{0.25,0.63,0.44}{{#1}}}
    \newcommand{\FloatTok}[1]{\textcolor[rgb]{0.25,0.63,0.44}{{#1}}}
    \newcommand{\CharTok}[1]{\textcolor[rgb]{0.25,0.44,0.63}{{#1}}}
    \newcommand{\StringTok}[1]{\textcolor[rgb]{0.25,0.44,0.63}{{#1}}}
    \newcommand{\CommentTok}[1]{\textcolor[rgb]{0.38,0.63,0.69}{\textit{{#1}}}}
    \newcommand{\OtherTok}[1]{\textcolor[rgb]{0.00,0.44,0.13}{{#1}}}
    \newcommand{\AlertTok}[1]{\textcolor[rgb]{1.00,0.00,0.00}{\textbf{{#1}}}}
    \newcommand{\FunctionTok}[1]{\textcolor[rgb]{0.02,0.16,0.49}{{#1}}}
    \newcommand{\RegionMarkerTok}[1]{{#1}}
    \newcommand{\ErrorTok}[1]{\textcolor[rgb]{1.00,0.00,0.00}{\textbf{{#1}}}}
    \newcommand{\NormalTok}[1]{{#1}}

    % Additional commands for more recent versions of Pandoc
    \newcommand{\ConstantTok}[1]{\textcolor[rgb]{0.53,0.00,0.00}{{#1}}}
    \newcommand{\SpecialCharTok}[1]{\textcolor[rgb]{0.25,0.44,0.63}{{#1}}}
    \newcommand{\VerbatimStringTok}[1]{\textcolor[rgb]{0.25,0.44,0.63}{{#1}}}
    \newcommand{\SpecialStringTok}[1]{\textcolor[rgb]{0.73,0.40,0.53}{{#1}}}
    \newcommand{\ImportTok}[1]{{#1}}
    \newcommand{\DocumentationTok}[1]{\textcolor[rgb]{0.73,0.13,0.13}{\textit{{#1}}}}
    \newcommand{\AnnotationTok}[1]{\textcolor[rgb]{0.38,0.63,0.69}{\textbf{\textit{{#1}}}}}
    \newcommand{\CommentVarTok}[1]{\textcolor[rgb]{0.38,0.63,0.69}{\textbf{\textit{{#1}}}}}
    \newcommand{\VariableTok}[1]{\textcolor[rgb]{0.10,0.09,0.49}{{#1}}}
    \newcommand{\ControlFlowTok}[1]{\textcolor[rgb]{0.00,0.44,0.13}{\textbf{{#1}}}}
    \newcommand{\OperatorTok}[1]{\textcolor[rgb]{0.40,0.40,0.40}{{#1}}}
    \newcommand{\BuiltInTok}[1]{{#1}}
    \newcommand{\ExtensionTok}[1]{{#1}}
    \newcommand{\PreprocessorTok}[1]{\textcolor[rgb]{0.74,0.48,0.00}{{#1}}}
    \newcommand{\AttributeTok}[1]{\textcolor[rgb]{0.49,0.56,0.16}{{#1}}}
    \newcommand{\InformationTok}[1]{\textcolor[rgb]{0.38,0.63,0.69}{\textbf{\textit{{#1}}}}}
    \newcommand{\WarningTok}[1]{\textcolor[rgb]{0.38,0.63,0.69}{\textbf{\textit{{#1}}}}}


    % Define a nice break command that doesn't care if a line doesn't already
    % exist.
    \def\br{\hspace*{\fill} \\* }
    % Math Jax compatibility definitions
    \def\gt{>}
    \def\lt{<}
    \let\Oldtex\TeX
    \let\Oldlatex\LaTeX
    \renewcommand{\TeX}{\textrm{\Oldtex}}
    \renewcommand{\LaTeX}{\textrm{\Oldlatex}}
    % Document parameters
    % Document title
    \title{laboratory3\_sample\_report-Copy1}
    
    
    
    
    
    
    
% Pygments definitions
\makeatletter
\def\PY@reset{\let\PY@it=\relax \let\PY@bf=\relax%
    \let\PY@ul=\relax \let\PY@tc=\relax%
    \let\PY@bc=\relax \let\PY@ff=\relax}
\def\PY@tok#1{\csname PY@tok@#1\endcsname}
\def\PY@toks#1+{\ifx\relax#1\empty\else%
    \PY@tok{#1}\expandafter\PY@toks\fi}
\def\PY@do#1{\PY@bc{\PY@tc{\PY@ul{%
    \PY@it{\PY@bf{\PY@ff{#1}}}}}}}
\def\PY#1#2{\PY@reset\PY@toks#1+\relax+\PY@do{#2}}

\@namedef{PY@tok@w}{\def\PY@tc##1{\textcolor[rgb]{0.73,0.73,0.73}{##1}}}
\@namedef{PY@tok@c}{\let\PY@it=\textit\def\PY@tc##1{\textcolor[rgb]{0.24,0.48,0.48}{##1}}}
\@namedef{PY@tok@cp}{\def\PY@tc##1{\textcolor[rgb]{0.61,0.40,0.00}{##1}}}
\@namedef{PY@tok@k}{\let\PY@bf=\textbf\def\PY@tc##1{\textcolor[rgb]{0.00,0.50,0.00}{##1}}}
\@namedef{PY@tok@kp}{\def\PY@tc##1{\textcolor[rgb]{0.00,0.50,0.00}{##1}}}
\@namedef{PY@tok@kt}{\def\PY@tc##1{\textcolor[rgb]{0.69,0.00,0.25}{##1}}}
\@namedef{PY@tok@o}{\def\PY@tc##1{\textcolor[rgb]{0.40,0.40,0.40}{##1}}}
\@namedef{PY@tok@ow}{\let\PY@bf=\textbf\def\PY@tc##1{\textcolor[rgb]{0.67,0.13,1.00}{##1}}}
\@namedef{PY@tok@nb}{\def\PY@tc##1{\textcolor[rgb]{0.00,0.50,0.00}{##1}}}
\@namedef{PY@tok@nf}{\def\PY@tc##1{\textcolor[rgb]{0.00,0.00,1.00}{##1}}}
\@namedef{PY@tok@nc}{\let\PY@bf=\textbf\def\PY@tc##1{\textcolor[rgb]{0.00,0.00,1.00}{##1}}}
\@namedef{PY@tok@nn}{\let\PY@bf=\textbf\def\PY@tc##1{\textcolor[rgb]{0.00,0.00,1.00}{##1}}}
\@namedef{PY@tok@ne}{\let\PY@bf=\textbf\def\PY@tc##1{\textcolor[rgb]{0.80,0.25,0.22}{##1}}}
\@namedef{PY@tok@nv}{\def\PY@tc##1{\textcolor[rgb]{0.10,0.09,0.49}{##1}}}
\@namedef{PY@tok@no}{\def\PY@tc##1{\textcolor[rgb]{0.53,0.00,0.00}{##1}}}
\@namedef{PY@tok@nl}{\def\PY@tc##1{\textcolor[rgb]{0.46,0.46,0.00}{##1}}}
\@namedef{PY@tok@ni}{\let\PY@bf=\textbf\def\PY@tc##1{\textcolor[rgb]{0.44,0.44,0.44}{##1}}}
\@namedef{PY@tok@na}{\def\PY@tc##1{\textcolor[rgb]{0.41,0.47,0.13}{##1}}}
\@namedef{PY@tok@nt}{\let\PY@bf=\textbf\def\PY@tc##1{\textcolor[rgb]{0.00,0.50,0.00}{##1}}}
\@namedef{PY@tok@nd}{\def\PY@tc##1{\textcolor[rgb]{0.67,0.13,1.00}{##1}}}
\@namedef{PY@tok@s}{\def\PY@tc##1{\textcolor[rgb]{0.73,0.13,0.13}{##1}}}
\@namedef{PY@tok@sd}{\let\PY@it=\textit\def\PY@tc##1{\textcolor[rgb]{0.73,0.13,0.13}{##1}}}
\@namedef{PY@tok@si}{\let\PY@bf=\textbf\def\PY@tc##1{\textcolor[rgb]{0.64,0.35,0.47}{##1}}}
\@namedef{PY@tok@se}{\let\PY@bf=\textbf\def\PY@tc##1{\textcolor[rgb]{0.67,0.36,0.12}{##1}}}
\@namedef{PY@tok@sr}{\def\PY@tc##1{\textcolor[rgb]{0.64,0.35,0.47}{##1}}}
\@namedef{PY@tok@ss}{\def\PY@tc##1{\textcolor[rgb]{0.10,0.09,0.49}{##1}}}
\@namedef{PY@tok@sx}{\def\PY@tc##1{\textcolor[rgb]{0.00,0.50,0.00}{##1}}}
\@namedef{PY@tok@m}{\def\PY@tc##1{\textcolor[rgb]{0.40,0.40,0.40}{##1}}}
\@namedef{PY@tok@gh}{\let\PY@bf=\textbf\def\PY@tc##1{\textcolor[rgb]{0.00,0.00,0.50}{##1}}}
\@namedef{PY@tok@gu}{\let\PY@bf=\textbf\def\PY@tc##1{\textcolor[rgb]{0.50,0.00,0.50}{##1}}}
\@namedef{PY@tok@gd}{\def\PY@tc##1{\textcolor[rgb]{0.63,0.00,0.00}{##1}}}
\@namedef{PY@tok@gi}{\def\PY@tc##1{\textcolor[rgb]{0.00,0.52,0.00}{##1}}}
\@namedef{PY@tok@gr}{\def\PY@tc##1{\textcolor[rgb]{0.89,0.00,0.00}{##1}}}
\@namedef{PY@tok@ge}{\let\PY@it=\textit}
\@namedef{PY@tok@gs}{\let\PY@bf=\textbf}
\@namedef{PY@tok@ges}{\let\PY@bf=\textbf\let\PY@it=\textit}
\@namedef{PY@tok@gp}{\let\PY@bf=\textbf\def\PY@tc##1{\textcolor[rgb]{0.00,0.00,0.50}{##1}}}
\@namedef{PY@tok@go}{\def\PY@tc##1{\textcolor[rgb]{0.44,0.44,0.44}{##1}}}
\@namedef{PY@tok@gt}{\def\PY@tc##1{\textcolor[rgb]{0.00,0.27,0.87}{##1}}}
\@namedef{PY@tok@err}{\def\PY@bc##1{{\setlength{\fboxsep}{\string -\fboxrule}\fcolorbox[rgb]{1.00,0.00,0.00}{1,1,1}{\strut ##1}}}}
\@namedef{PY@tok@kc}{\let\PY@bf=\textbf\def\PY@tc##1{\textcolor[rgb]{0.00,0.50,0.00}{##1}}}
\@namedef{PY@tok@kd}{\let\PY@bf=\textbf\def\PY@tc##1{\textcolor[rgb]{0.00,0.50,0.00}{##1}}}
\@namedef{PY@tok@kn}{\let\PY@bf=\textbf\def\PY@tc##1{\textcolor[rgb]{0.00,0.50,0.00}{##1}}}
\@namedef{PY@tok@kr}{\let\PY@bf=\textbf\def\PY@tc##1{\textcolor[rgb]{0.00,0.50,0.00}{##1}}}
\@namedef{PY@tok@bp}{\def\PY@tc##1{\textcolor[rgb]{0.00,0.50,0.00}{##1}}}
\@namedef{PY@tok@fm}{\def\PY@tc##1{\textcolor[rgb]{0.00,0.00,1.00}{##1}}}
\@namedef{PY@tok@vc}{\def\PY@tc##1{\textcolor[rgb]{0.10,0.09,0.49}{##1}}}
\@namedef{PY@tok@vg}{\def\PY@tc##1{\textcolor[rgb]{0.10,0.09,0.49}{##1}}}
\@namedef{PY@tok@vi}{\def\PY@tc##1{\textcolor[rgb]{0.10,0.09,0.49}{##1}}}
\@namedef{PY@tok@vm}{\def\PY@tc##1{\textcolor[rgb]{0.10,0.09,0.49}{##1}}}
\@namedef{PY@tok@sa}{\def\PY@tc##1{\textcolor[rgb]{0.73,0.13,0.13}{##1}}}
\@namedef{PY@tok@sb}{\def\PY@tc##1{\textcolor[rgb]{0.73,0.13,0.13}{##1}}}
\@namedef{PY@tok@sc}{\def\PY@tc##1{\textcolor[rgb]{0.73,0.13,0.13}{##1}}}
\@namedef{PY@tok@dl}{\def\PY@tc##1{\textcolor[rgb]{0.73,0.13,0.13}{##1}}}
\@namedef{PY@tok@s2}{\def\PY@tc##1{\textcolor[rgb]{0.73,0.13,0.13}{##1}}}
\@namedef{PY@tok@sh}{\def\PY@tc##1{\textcolor[rgb]{0.73,0.13,0.13}{##1}}}
\@namedef{PY@tok@s1}{\def\PY@tc##1{\textcolor[rgb]{0.73,0.13,0.13}{##1}}}
\@namedef{PY@tok@mb}{\def\PY@tc##1{\textcolor[rgb]{0.40,0.40,0.40}{##1}}}
\@namedef{PY@tok@mf}{\def\PY@tc##1{\textcolor[rgb]{0.40,0.40,0.40}{##1}}}
\@namedef{PY@tok@mh}{\def\PY@tc##1{\textcolor[rgb]{0.40,0.40,0.40}{##1}}}
\@namedef{PY@tok@mi}{\def\PY@tc##1{\textcolor[rgb]{0.40,0.40,0.40}{##1}}}
\@namedef{PY@tok@il}{\def\PY@tc##1{\textcolor[rgb]{0.40,0.40,0.40}{##1}}}
\@namedef{PY@tok@mo}{\def\PY@tc##1{\textcolor[rgb]{0.40,0.40,0.40}{##1}}}
\@namedef{PY@tok@ch}{\let\PY@it=\textit\def\PY@tc##1{\textcolor[rgb]{0.24,0.48,0.48}{##1}}}
\@namedef{PY@tok@cm}{\let\PY@it=\textit\def\PY@tc##1{\textcolor[rgb]{0.24,0.48,0.48}{##1}}}
\@namedef{PY@tok@cpf}{\let\PY@it=\textit\def\PY@tc##1{\textcolor[rgb]{0.24,0.48,0.48}{##1}}}
\@namedef{PY@tok@c1}{\let\PY@it=\textit\def\PY@tc##1{\textcolor[rgb]{0.24,0.48,0.48}{##1}}}
\@namedef{PY@tok@cs}{\let\PY@it=\textit\def\PY@tc##1{\textcolor[rgb]{0.24,0.48,0.48}{##1}}}

\def\PYZbs{\char`\\}
\def\PYZus{\char`\_}
\def\PYZob{\char`\{}
\def\PYZcb{\char`\}}
\def\PYZca{\char`\^}
\def\PYZam{\char`\&}
\def\PYZlt{\char`\<}
\def\PYZgt{\char`\>}
\def\PYZsh{\char`\#}
\def\PYZpc{\char`\%}
\def\PYZdl{\char`\$}
\def\PYZhy{\char`\-}
\def\PYZsq{\char`\'}
\def\PYZdq{\char`\"}
\def\PYZti{\char`\~}
% for compatibility with earlier versions
\def\PYZat{@}
\def\PYZlb{[}
\def\PYZrb{]}
\makeatother


    % For linebreaks inside Verbatim environment from package fancyvrb.
    \makeatletter
        \newbox\Wrappedcontinuationbox
        \newbox\Wrappedvisiblespacebox
        \newcommand*\Wrappedvisiblespace {\textcolor{red}{\textvisiblespace}}
        \newcommand*\Wrappedcontinuationsymbol {\textcolor{red}{\llap{\tiny$\m@th\hookrightarrow$}}}
        \newcommand*\Wrappedcontinuationindent {3ex }
        \newcommand*\Wrappedafterbreak {\kern\Wrappedcontinuationindent\copy\Wrappedcontinuationbox}
        % Take advantage of the already applied Pygments mark-up to insert
        % potential linebreaks for TeX processing.
        %        {, <, #, %, $, ' and ": go to next line.
        %        _, }, ^, &, >, - and ~: stay at end of broken line.
        % Use of \textquotesingle for straight quote.
        \newcommand*\Wrappedbreaksatspecials {%
            \def\PYGZus{\discretionary{\char`\_}{\Wrappedafterbreak}{\char`\_}}%
            \def\PYGZob{\discretionary{}{\Wrappedafterbreak\char`\{}{\char`\{}}%
            \def\PYGZcb{\discretionary{\char`\}}{\Wrappedafterbreak}{\char`\}}}%
            \def\PYGZca{\discretionary{\char`\^}{\Wrappedafterbreak}{\char`\^}}%
            \def\PYGZam{\discretionary{\char`\&}{\Wrappedafterbreak}{\char`\&}}%
            \def\PYGZlt{\discretionary{}{\Wrappedafterbreak\char`\<}{\char`\<}}%
            \def\PYGZgt{\discretionary{\char`\>}{\Wrappedafterbreak}{\char`\>}}%
            \def\PYGZsh{\discretionary{}{\Wrappedafterbreak\char`\#}{\char`\#}}%
            \def\PYGZpc{\discretionary{}{\Wrappedafterbreak\char`\%}{\char`\%}}%
            \def\PYGZdl{\discretionary{}{\Wrappedafterbreak\char`\$}{\char`\$}}%
            \def\PYGZhy{\discretionary{\char`\-}{\Wrappedafterbreak}{\char`\-}}%
            \def\PYGZsq{\discretionary{}{\Wrappedafterbreak\textquotesingle}{\textquotesingle}}%
            \def\PYGZdq{\discretionary{}{\Wrappedafterbreak\char`\"}{\char`\"}}%
            \def\PYGZti{\discretionary{\char`\~}{\Wrappedafterbreak}{\char`\~}}%
        }
        % Some characters . , ; ? ! / are not pygmentized.
        % This macro makes them "active" and they will insert potential linebreaks
        \newcommand*\Wrappedbreaksatpunct {%
            \lccode`\~`\.\lowercase{\def~}{\discretionary{\hbox{\char`\.}}{\Wrappedafterbreak}{\hbox{\char`\.}}}%
            \lccode`\~`\,\lowercase{\def~}{\discretionary{\hbox{\char`\,}}{\Wrappedafterbreak}{\hbox{\char`\,}}}%
            \lccode`\~`\;\lowercase{\def~}{\discretionary{\hbox{\char`\;}}{\Wrappedafterbreak}{\hbox{\char`\;}}}%
            \lccode`\~`\:\lowercase{\def~}{\discretionary{\hbox{\char`\:}}{\Wrappedafterbreak}{\hbox{\char`\:}}}%
            \lccode`\~`\?\lowercase{\def~}{\discretionary{\hbox{\char`\?}}{\Wrappedafterbreak}{\hbox{\char`\?}}}%
            \lccode`\~`\!\lowercase{\def~}{\discretionary{\hbox{\char`\!}}{\Wrappedafterbreak}{\hbox{\char`\!}}}%
            \lccode`\~`\/\lowercase{\def~}{\discretionary{\hbox{\char`\/}}{\Wrappedafterbreak}{\hbox{\char`\/}}}%
            \catcode`\.\active
            \catcode`\,\active
            \catcode`\;\active
            \catcode`\:\active
            \catcode`\?\active
            \catcode`\!\active
            \catcode`\/\active
            \lccode`\~`\~
        }
    \makeatother

    \let\OriginalVerbatim=\Verbatim
    \makeatletter
    \renewcommand{\Verbatim}[1][1]{%
        %\parskip\z@skip
        \sbox\Wrappedcontinuationbox {\Wrappedcontinuationsymbol}%
        \sbox\Wrappedvisiblespacebox {\FV@SetupFont\Wrappedvisiblespace}%
        \def\FancyVerbFormatLine ##1{\hsize\linewidth
            \vtop{\raggedright\hyphenpenalty\z@\exhyphenpenalty\z@
                \doublehyphendemerits\z@\finalhyphendemerits\z@
                \strut ##1\strut}%
        }%
        % If the linebreak is at a space, the latter will be displayed as visible
        % space at end of first line, and a continuation symbol starts next line.
        % Stretch/shrink are however usually zero for typewriter font.
        \def\FV@Space {%
            \nobreak\hskip\z@ plus\fontdimen3\font minus\fontdimen4\font
            \discretionary{\copy\Wrappedvisiblespacebox}{\Wrappedafterbreak}
            {\kern\fontdimen2\font}%
        }%

        % Allow breaks at special characters using \PYG... macros.
        \Wrappedbreaksatspecials
        % Breaks at punctuation characters . , ; ? ! and / need catcode=\active
        \OriginalVerbatim[#1,codes*=\Wrappedbreaksatpunct]%
    }
    \makeatother

    % Exact colors from NB
    \definecolor{incolor}{HTML}{303F9F}
    \definecolor{outcolor}{HTML}{D84315}
    \definecolor{cellborder}{HTML}{CFCFCF}
    \definecolor{cellbackground}{HTML}{F7F7F7}

    % prompt
    \makeatletter
    \newcommand{\boxspacing}{\kern\kvtcb@left@rule\kern\kvtcb@boxsep}
    \makeatother
    \newcommand{\prompt}[4]{
        {\ttfamily\llap{{\color{#2}[#3]:\hspace{3pt}#4}}\vspace{-\baselineskip}}
    }
    

    
    % Prevent overflowing lines due to hard-to-break entities
    \sloppy
    % Setup hyperref package
    \hypersetup{
      breaklinks=true,  % so long urls are correctly broken across lines
      colorlinks=true,
      urlcolor=urlcolor,
      linkcolor=linkcolor,
      citecolor=citecolor,
      }
    % Slightly bigger margins than the latex defaults
    
    \geometry{verbose,tmargin=1in,bmargin=1in,lmargin=1in,rmargin=1in}
    
    

\begin{document}
    
    \maketitle
    
    

    
    \hypertarget{laboratory-3-sample-report}{%
\section{Laboratory 3 Sample Report}\label{laboratory-3-sample-report}}

\hypertarget{flow-measurement}{%
\subsection{Flow Measurement}\label{flow-measurement}}

\hypertarget{introduction}{%
\subsubsection{1. Introduction}\label{introduction}}

The purpose of this laboratory exercise is to determine the flow rate of
an incompressible fluid using various flow measurement devices,
including Venturi meter, orifice plate meter, and rotameter, and to
evaluate their accuracy and suitability for subsequent measurements.

    \hypertarget{experimental-method}{%
\subsubsection{2. Experimental Method}\label{experimental-method}}

\hypertarget{apparatus-and-materials}{%
\paragraph{Apparatus and Materials}\label{apparatus-and-materials}}

\begin{itemize}
\tightlist
\item
  Hydraulic bench with adjustable flow control
\item
  Venturi meter
\item
  Orifice plate meter
\item
  Rotameter
\item
  Stopwatch
\item
  Measuring beaker
\item
  Manometer tubes
\item
  Raspberry Pi Zero-W computer
\item
  Mass Flow Meter
\item
  Safety goggles
\end{itemize}

\hypertarget{procedure}{%
\paragraph{Procedure}\label{procedure}}

\begin{enumerate}
\def\labelenumi{\arabic{enumi}.}
\tightlist
\item
  \textbf{Setup Preparation}:

  \begin{itemize}
  \tightlist
  \item
    Close all valves and ensure the hydraulic bench pump is turned off.
  \item
    Connect the Venturi meter, orifice plate meter, and rotameter to the
    hydraulic bench in series.
  \end{itemize}
\item
  \textbf{Flow Initialization}:

  \begin{itemize}
  \tightlist
  \item
    Open the inlet valve gradually.
  \item
    Start the pump and adjust the discharge valve until the rotameter
    indicates a reading of approximately 10 mm.
  \end{itemize}
\item
  \textbf{Data Collection}:

  \begin{itemize}
  \tightlist
  \item
    Use the beaker and stopwatch to measure volumetric discharge at the
    outlet. Usually record time to fill the beaker, repeat several times
    (at least 3), then compute the average time to fill the beaker.
  \item
    Record the rotameter and manometer readings for each flow condition.
  \item
    Incrementally increase the rotameter reading to approximately 220
    mm, recording at least five measurements.
  \item
    Connect to the Raspberry Pi computer, find and start the measurement
    program (the instructor will usually get this working in advance as
    well as plumb the mass flow meter into the system); you can either
    program it to read meter counts or liters.
  \item
    Collect adequate readings to populate the data tables below (be sure
    you have 5 different flow rates)
  \end{itemize}
\item
  \textbf{Shutdown}:

  \begin{itemize}
  \tightlist
  \item
    Stop the mass flow meter program (do an elegant exit from the
    program).
  \item
    Turn off the Raspberry Pi (from the console type ``sudo shutdown -h
    now'')
  \item
    Close the discharge valve, then the inlet valve, and turn off the
    pump.
  \end{itemize}
\end{enumerate}

\hypertarget{safety-considerations}{%
\subsubsection{Safety Considerations}\label{safety-considerations}}

\begin{itemize}
\tightlist
\item
  Always wear safety goggles when handling pressurized systems and
  adding air to manometers.
\item
  Ensure that the bench and surrounding area are free of obstructions to
  avoid tripping hazards.
\item
  Follow TA instructions when troubleshooting equipment, especially when
  dealing with flooded manometer tubes.
\end{itemize}

    \hypertarget{results}{%
\subsubsection{3. Results}\label{results}}

The table below is the actual data collected for five (5) different flow
settings. All the values are direct readings except for the
time-to-fill, which are arithmetic means from three repeated
measurements.

\textbf{Notes}

\begin{enumerate}
\def\labelenumi{\arabic{enumi}.}
\tightlist
\item
  Rotometer outlet heads were not read.
\item
  Time to fill is arithmetic mean of three readings.
\end{enumerate}

\begin{longtable}[]{@{}llrrrrr@{}}
\toprule
\begin{minipage}[b]{0.27\columnwidth}\raggedright
Quantity\strut
\end{minipage} & \begin{minipage}[b]{0.19\columnwidth}\raggedright
Device\strut
\end{minipage} & \begin{minipage}[b]{0.07\columnwidth}\raggedleft
Trial 1\strut
\end{minipage} & \begin{minipage}[b]{0.07\columnwidth}\raggedleft
Trial 2\strut
\end{minipage} & \begin{minipage}[b]{0.07\columnwidth}\raggedleft
Trial 3\strut
\end{minipage} & \begin{minipage}[b]{0.07\columnwidth}\raggedleft
Trial 4\strut
\end{minipage} & \begin{minipage}[b]{0.07\columnwidth}\raggedleft
Trial 5\strut
\end{minipage}\tabularnewline
\midrule
\endhead
\begin{minipage}[t]{0.27\columnwidth}\raggedright
\strut
\end{minipage} & \begin{minipage}[t]{0.19\columnwidth}\raggedright
Venturi-In\strut
\end{minipage} & \begin{minipage}[t]{0.07\columnwidth}\raggedleft
158\strut
\end{minipage} & \begin{minipage}[t]{0.07\columnwidth}\raggedleft
210\strut
\end{minipage} & \begin{minipage}[t]{0.07\columnwidth}\raggedleft
170\strut
\end{minipage} & \begin{minipage}[t]{0.07\columnwidth}\raggedleft
214\strut
\end{minipage} & \begin{minipage}[t]{0.07\columnwidth}\raggedleft
310\strut
\end{minipage}\tabularnewline
\begin{minipage}[t]{0.27\columnwidth}\raggedright
\strut
\end{minipage} & \begin{minipage}[t]{0.19\columnwidth}\raggedright
Venturi-Out\strut
\end{minipage} & \begin{minipage}[t]{0.07\columnwidth}\raggedleft
110\strut
\end{minipage} & \begin{minipage}[t]{0.07\columnwidth}\raggedleft
132\strut
\end{minipage} & \begin{minipage}[t]{0.07\columnwidth}\raggedleft
80\strut
\end{minipage} & \begin{minipage}[t]{0.07\columnwidth}\raggedleft
90\strut
\end{minipage} & \begin{minipage}[t]{0.07\columnwidth}\raggedleft
302\strut
\end{minipage}\tabularnewline
\begin{minipage}[t]{0.27\columnwidth}\raggedright
\strut
\end{minipage} & \begin{minipage}[t]{0.19\columnwidth}\raggedright
Expansion-In\strut
\end{minipage} & \begin{minipage}[t]{0.07\columnwidth}\raggedleft
150\strut
\end{minipage} & \begin{minipage}[t]{0.07\columnwidth}\raggedleft
196\strut
\end{minipage} & \begin{minipage}[t]{0.07\columnwidth}\raggedleft
150\strut
\end{minipage} & \begin{minipage}[t]{0.07\columnwidth}\raggedleft
192\strut
\end{minipage} & \begin{minipage}[t]{0.07\columnwidth}\raggedleft
306\strut
\end{minipage}\tabularnewline
\begin{minipage}[t]{0.27\columnwidth}\raggedright
\strut
\end{minipage} & \begin{minipage}[t]{0.19\columnwidth}\raggedright
Expansion-Out\strut
\end{minipage} & \begin{minipage}[t]{0.07\columnwidth}\raggedleft
170\strut
\end{minipage} & \begin{minipage}[t]{0.07\columnwidth}\raggedleft
192\strut
\end{minipage} & \begin{minipage}[t]{0.07\columnwidth}\raggedleft
170\strut
\end{minipage} & \begin{minipage}[t]{0.07\columnwidth}\raggedleft
192\strut
\end{minipage} & \begin{minipage}[t]{0.07\columnwidth}\raggedleft
302\strut
\end{minipage}\tabularnewline
\begin{minipage}[t]{0.27\columnwidth}\raggedright
Manometer Reading\strut
\end{minipage} & \begin{minipage}[t]{0.19\columnwidth}\raggedright
Orifice-In\strut
\end{minipage} & \begin{minipage}[t]{0.07\columnwidth}\raggedleft
152\strut
\end{minipage} & \begin{minipage}[t]{0.07\columnwidth}\raggedleft
210\strut
\end{minipage} & \begin{minipage}[t]{0.07\columnwidth}\raggedleft
162\strut
\end{minipage} & \begin{minipage}[t]{0.07\columnwidth}\raggedleft
220\strut
\end{minipage} & \begin{minipage}[t]{0.07\columnwidth}\raggedleft
306\strut
\end{minipage}\tabularnewline
\begin{minipage}[t]{0.27\columnwidth}\raggedright
(mm of water)\strut
\end{minipage} & \begin{minipage}[t]{0.19\columnwidth}\raggedright
Orfice-Out\strut
\end{minipage} & \begin{minipage}[t]{0.07\columnwidth}\raggedleft
108\strut
\end{minipage} & \begin{minipage}[t]{0.07\columnwidth}\raggedleft
126\strut
\end{minipage} & \begin{minipage}[t]{0.07\columnwidth}\raggedleft
70\strut
\end{minipage} & \begin{minipage}[t]{0.07\columnwidth}\raggedleft
78\strut
\end{minipage} & \begin{minipage}[t]{0.07\columnwidth}\raggedleft
302\strut
\end{minipage}\tabularnewline
\begin{minipage}[t]{0.27\columnwidth}\raggedright
\strut
\end{minipage} & \begin{minipage}[t]{0.19\columnwidth}\raggedright
Elbow-In\strut
\end{minipage} & \begin{minipage}[t]{0.07\columnwidth}\raggedleft
118\strut
\end{minipage} & \begin{minipage}[t]{0.07\columnwidth}\raggedleft
140\strut
\end{minipage} & \begin{minipage}[t]{0.07\columnwidth}\raggedleft
88\strut
\end{minipage} & \begin{minipage}[t]{0.07\columnwidth}\raggedleft
100\strut
\end{minipage} & \begin{minipage}[t]{0.07\columnwidth}\raggedleft
302\strut
\end{minipage}\tabularnewline
\begin{minipage}[t]{0.27\columnwidth}\raggedright
\strut
\end{minipage} & \begin{minipage}[t]{0.19\columnwidth}\raggedright
Elbow-Out\strut
\end{minipage} & \begin{minipage}[t]{0.07\columnwidth}\raggedleft
117\strut
\end{minipage} & \begin{minipage}[t]{0.07\columnwidth}\raggedleft
138\strut
\end{minipage} & \begin{minipage}[t]{0.07\columnwidth}\raggedleft
84\strut
\end{minipage} & \begin{minipage}[t]{0.07\columnwidth}\raggedleft
96\strut
\end{minipage} & \begin{minipage}[t]{0.07\columnwidth}\raggedleft
303\strut
\end{minipage}\tabularnewline
\begin{minipage}[t]{0.27\columnwidth}\raggedright
\strut
\end{minipage} & \begin{minipage}[t]{0.19\columnwidth}\raggedright
Rotometer-Out\strut
\end{minipage} & \begin{minipage}[t]{0.07\columnwidth}\raggedleft
--\strut
\end{minipage} & \begin{minipage}[t]{0.07\columnwidth}\raggedleft
--\strut
\end{minipage} & \begin{minipage}[t]{0.07\columnwidth}\raggedleft
--\strut
\end{minipage} & \begin{minipage}[t]{0.07\columnwidth}\raggedleft
--\strut
\end{minipage} & \begin{minipage}[t]{0.07\columnwidth}\raggedleft
--\strut
\end{minipage}\tabularnewline
\begin{minipage}[t]{0.27\columnwidth}\raggedright
\_\_\_\_\_\_\_\_\_\_\_\_\_\_\_\_\_\_\_\_\_\_\_\_\strut
\end{minipage} & \begin{minipage}[t]{0.19\columnwidth}\raggedright
\_\_\_\_\_\_\_\_\_\_\_\_\_\_\_\_\_\strut
\end{minipage} & \begin{minipage}[t]{0.07\columnwidth}\raggedleft
\_\_\_\_\_\_\strut
\end{minipage} & \begin{minipage}[t]{0.07\columnwidth}\raggedleft
\_\_\_\_\_\_\_\strut
\end{minipage} & \begin{minipage}[t]{0.07\columnwidth}\raggedleft
\_\_\_\_\_\_\_\strut
\end{minipage} & \begin{minipage}[t]{0.07\columnwidth}\raggedleft
\_\_\_\_\_\_\_\strut
\end{minipage} & \begin{minipage}[t]{0.07\columnwidth}\raggedleft
\_\_\_\_\_\_\_\strut
\end{minipage}\tabularnewline
\begin{minipage}[t]{0.27\columnwidth}\raggedright
Time to Fill (sec)\strut
\end{minipage} & \begin{minipage}[t]{0.19\columnwidth}\raggedright
Beaker (2L)\strut
\end{minipage} & \begin{minipage}[t]{0.07\columnwidth}\raggedleft
12.49\strut
\end{minipage} & \begin{minipage}[t]{0.07\columnwidth}\raggedleft
9.06\strut
\end{minipage} & \begin{minipage}[t]{0.07\columnwidth}\raggedleft
7.6\strut
\end{minipage} & \begin{minipage}[t]{0.07\columnwidth}\raggedleft
6.71\strut
\end{minipage} & \begin{minipage}[t]{0.07\columnwidth}\raggedleft
27.55\strut
\end{minipage}\tabularnewline
\begin{minipage}[t]{0.27\columnwidth}\raggedright
Counts per second\strut
\end{minipage} & \begin{minipage}[t]{0.19\columnwidth}\raggedright
Mass Flow Meter\strut
\end{minipage} & \begin{minipage}[t]{0.07\columnwidth}\raggedleft
85\strut
\end{minipage} & \begin{minipage}[t]{0.07\columnwidth}\raggedleft
113\strut
\end{minipage} & \begin{minipage}[t]{0.07\columnwidth}\raggedleft
122\strut
\end{minipage} & \begin{minipage}[t]{0.07\columnwidth}\raggedleft
146\strut
\end{minipage} & \begin{minipage}[t]{0.07\columnwidth}\raggedleft
31\strut
\end{minipage}\tabularnewline
\begin{minipage}[t]{0.27\columnwidth}\raggedright
Plummet height (mm)\strut
\end{minipage} & \begin{minipage}[t]{0.19\columnwidth}\raggedright
Rotameter\strut
\end{minipage} & \begin{minipage}[t]{0.07\columnwidth}\raggedleft
75\strut
\end{minipage} & \begin{minipage}[t]{0.07\columnwidth}\raggedleft
100\strut
\end{minipage} & \begin{minipage}[t]{0.07\columnwidth}\raggedleft
110\strut
\end{minipage} & \begin{minipage}[t]{0.07\columnwidth}\raggedleft
130\strut
\end{minipage} & \begin{minipage}[t]{0.07\columnwidth}\raggedleft
20\strut
\end{minipage}\tabularnewline
\begin{minipage}[t]{0.27\columnwidth}\raggedright
\_\_\_\_\_\_\_\_\_\_\_\_\_\_\_\_\_\_\_\_\_\_\_\_\strut
\end{minipage} & \begin{minipage}[t]{0.19\columnwidth}\raggedright
\_\_\_\_\_\_\_\_\_\_\_\_\_\_\_\_\_\strut
\end{minipage} & \begin{minipage}[t]{0.07\columnwidth}\raggedleft
\_\_\_\_\_\_\strut
\end{minipage} & \begin{minipage}[t]{0.07\columnwidth}\raggedleft
\_\_\_\_\_\_\_\strut
\end{minipage} & \begin{minipage}[t]{0.07\columnwidth}\raggedleft
\_\_\_\_\_\_\_\strut
\end{minipage} & \begin{minipage}[t]{0.07\columnwidth}\raggedleft
\_\_\_\_\_\_\_\strut
\end{minipage} & \begin{minipage}[t]{0.07\columnwidth}\raggedleft
\_\_\_\_\_\_\_\strut
\end{minipage}\tabularnewline
\bottomrule
\end{longtable}

Formulas employed

\begin{enumerate}
\def\labelenumi{\arabic{enumi}.}
\tightlist
\item
  Venturi: \$Q =C\_\{venturi\}
  A\_2\textsubscript{(\frac{2g}{1-\frac{A_2^2}{A_1^2}})}\sqrt{\frac{P_1}{\rho g}-\frac{P_2}{\rho g}}
  \$
\item
  Expansion: \$Q = C\_\{expansion\}
  \frac{\sqrt{2g(h_1-h_2)}}{\frac{1}{A_1}-\frac{1}{A_2}} \$
\item
  Orifice: \$Q = C\_\{orifice\}
  A\_u\textsubscript{(\frac{2g}{1-\frac{A_d^2}{A_u^2}})}\sqrt{\frac{P_u}{\rho g}-\frac{P_d}{\rho g}}
  \$
\item
  Flow Meter: \$Q = C\_\{meter\}
  \frac{\text{count}}{\text{dwell interval}} \$
\end{enumerate}

Scripts to accomplish the calculations are included below.

The table below is populated with calculated values. The flow rates are
computed using the formulas listed above with meter constants set to
1.0. Once these flows are computed the appropriate meter constant is
computed from the \emph{true} flow, which is the value determined using
the beaker and stopwatch. These meter constants appear in the following
table.

\textbf{Notes}

\begin{enumerate}
\def\labelenumi{\arabic{enumi}.}
\tightlist
\item
  Flows in table are with meter constant of unity.
\item
  Elbow not useful to meter flow, \(\Delta h\) values are too small
  (unreadable)
\end{enumerate}

\begin{longtable}[]{@{}llrrrrr@{}}
\toprule
\begin{minipage}[b]{0.27\columnwidth}\raggedright
Quantity\strut
\end{minipage} & \begin{minipage}[b]{0.19\columnwidth}\raggedright
Device\strut
\end{minipage} & \begin{minipage}[b]{0.07\columnwidth}\raggedleft
Trial 1\strut
\end{minipage} & \begin{minipage}[b]{0.07\columnwidth}\raggedleft
Trial 2\strut
\end{minipage} & \begin{minipage}[b]{0.07\columnwidth}\raggedleft
Trial 3\strut
\end{minipage} & \begin{minipage}[b]{0.07\columnwidth}\raggedleft
Trial 4\strut
\end{minipage} & \begin{minipage}[b]{0.07\columnwidth}\raggedleft
Trial 5\strut
\end{minipage}\tabularnewline
\midrule
\endhead
\begin{minipage}[t]{0.27\columnwidth}\raggedright
Volume Flow (l/sec)\strut
\end{minipage} & \begin{minipage}[t]{0.19\columnwidth}\raggedright
Beaker\strut
\end{minipage} & \begin{minipage}[t]{0.07\columnwidth}\raggedleft
0.1601\strut
\end{minipage} & \begin{minipage}[t]{0.07\columnwidth}\raggedleft
0.2208\strut
\end{minipage} & \begin{minipage}[t]{0.07\columnwidth}\raggedleft
0.2632\strut
\end{minipage} & \begin{minipage}[t]{0.07\columnwidth}\raggedleft
0.2981\strut
\end{minipage} & \begin{minipage}[t]{0.07\columnwidth}\raggedleft
0.0726\strut
\end{minipage}\tabularnewline
\begin{minipage}[t]{0.27\columnwidth}\raggedright
Volume Flow (l/sec)\strut
\end{minipage} & \begin{minipage}[t]{0.19\columnwidth}\raggedright
Venturi\strut
\end{minipage} & \begin{minipage}[t]{0.07\columnwidth}\raggedleft
1.0082\strut
\end{minipage} & \begin{minipage}[t]{0.07\columnwidth}\raggedleft
1.2852\strut
\end{minipage} & \begin{minipage}[t]{0.07\columnwidth}\raggedleft
1.3806\strut
\end{minipage} & \begin{minipage}[t]{0.07\columnwidth}\raggedleft
1.6205\strut
\end{minipage} & \begin{minipage}[t]{0.07\columnwidth}\raggedleft
0.4116\strut
\end{minipage}\tabularnewline
\begin{minipage}[t]{0.27\columnwidth}\raggedright
Volume Flow (l/sec)\strut
\end{minipage} & \begin{minipage}[t]{0.19\columnwidth}\raggedright
Expansion\strut
\end{minipage} & \begin{minipage}[t]{0.07\columnwidth}\raggedleft
0.4437\strut
\end{minipage} & \begin{minipage}[t]{0.07\columnwidth}\raggedleft
0.1984\strut
\end{minipage} & \begin{minipage}[t]{0.07\columnwidth}\raggedleft
0.4437\strut
\end{minipage} & \begin{minipage}[t]{0.07\columnwidth}\raggedleft
0.1403\strut
\end{minipage} & \begin{minipage}[t]{0.07\columnwidth}\raggedleft
0.1984\strut
\end{minipage}\tabularnewline
\begin{minipage}[t]{0.27\columnwidth}\raggedright
Volume Flow (l/sec)\strut
\end{minipage} & \begin{minipage}[t]{0.19\columnwidth}\raggedright
Orifice\strut
\end{minipage} & \begin{minipage}[t]{0.07\columnwidth}\raggedleft
8.8957\strut
\end{minipage} & \begin{minipage}[t]{0.07\columnwidth}\raggedleft
12.2911\strut
\end{minipage} & \begin{minipage}[t]{0.07\columnwidth}\raggedleft
12.8631\strut
\end{minipage} & \begin{minipage}[t]{0.07\columnwidth}\raggedleft
15.9807\strut
\end{minipage} & \begin{minipage}[t]{0.07\columnwidth}\raggedleft
2.6821\strut
\end{minipage}\tabularnewline
\begin{minipage}[t]{0.27\columnwidth}\raggedright
Volume Flow (l/sec)\strut
\end{minipage} & \begin{minipage}[t]{0.19\columnwidth}\raggedright
Elbow\strut
\end{minipage} & \begin{minipage}[t]{0.07\columnwidth}\raggedleft
--\strut
\end{minipage} & \begin{minipage}[t]{0.07\columnwidth}\raggedleft
--\strut
\end{minipage} & \begin{minipage}[t]{0.07\columnwidth}\raggedleft
--\strut
\end{minipage} & \begin{minipage}[t]{0.07\columnwidth}\raggedleft
--\strut
\end{minipage} & \begin{minipage}[t]{0.07\columnwidth}\raggedleft
--\strut
\end{minipage}\tabularnewline
\begin{minipage}[t]{0.27\columnwidth}\raggedright
\_\_\_\_\_\_\_\_\_\_\_\_\_\_\_\_\_\_\_\_\_\_\_\_\strut
\end{minipage} & \begin{minipage}[t]{0.19\columnwidth}\raggedright
\_\_\_\_\_\_\_\_\_\_\_\_\_\_\_\_\_\strut
\end{minipage} & \begin{minipage}[t]{0.07\columnwidth}\raggedleft
\_\_\_\_\_\_\strut
\end{minipage} & \begin{minipage}[t]{0.07\columnwidth}\raggedleft
\_\_\_\_\_\_\_\strut
\end{minipage} & \begin{minipage}[t]{0.07\columnwidth}\raggedleft
\_\_\_\_\_\_\_\strut
\end{minipage} & \begin{minipage}[t]{0.07\columnwidth}\raggedleft
\_\_\_\_\_\_\_\strut
\end{minipage} & \begin{minipage}[t]{0.07\columnwidth}\raggedleft
\_\_\_\_\_\_\_\strut
\end{minipage}\tabularnewline
\bottomrule
\end{longtable}

The table below is a listing of the meter constants for the various
devices.

\textbf{Notes}

\begin{enumerate}
\def\labelenumi{\arabic{enumi}.}
\tightlist
\item
  Divide Beaker flow (above) by device flow to determine meter constant;
  use to populate table below.
\end{enumerate}

\begin{longtable}[]{@{}llrrrrr@{}}
\toprule
\begin{minipage}[b]{0.27\columnwidth}\raggedright
Quantity\strut
\end{minipage} & \begin{minipage}[b]{0.19\columnwidth}\raggedright
Device\strut
\end{minipage} & \begin{minipage}[b]{0.07\columnwidth}\raggedleft
Trial 1\strut
\end{minipage} & \begin{minipage}[b]{0.07\columnwidth}\raggedleft
Trial 2\strut
\end{minipage} & \begin{minipage}[b]{0.07\columnwidth}\raggedleft
Trial 3\strut
\end{minipage} & \begin{minipage}[b]{0.07\columnwidth}\raggedleft
Trial 4\strut
\end{minipage} & \begin{minipage}[b]{0.07\columnwidth}\raggedleft
Trial 5\strut
\end{minipage}\tabularnewline
\midrule
\endhead
\begin{minipage}[t]{0.27\columnwidth}\raggedright
Meter Constant\strut
\end{minipage} & \begin{minipage}[t]{0.19\columnwidth}\raggedright
Rotameter\strut
\end{minipage} & \begin{minipage}[t]{0.07\columnwidth}\raggedleft
0.0021\strut
\end{minipage} & \begin{minipage}[t]{0.07\columnwidth}\raggedleft
0.0022\strut
\end{minipage} & \begin{minipage}[t]{0.07\columnwidth}\raggedleft
0.0024\strut
\end{minipage} & \begin{minipage}[t]{0.07\columnwidth}\raggedleft
0.0023\strut
\end{minipage} & \begin{minipage}[t]{0.07\columnwidth}\raggedleft
0.0036\strut
\end{minipage}\tabularnewline
\begin{minipage}[t]{0.27\columnwidth}\raggedright
Meter Constant\strut
\end{minipage} & \begin{minipage}[t]{0.19\columnwidth}\raggedright
Venturi\strut
\end{minipage} & \begin{minipage}[t]{0.07\columnwidth}\raggedleft
0.1587\strut
\end{minipage} & \begin{minipage}[t]{0.07\columnwidth}\raggedleft
0.1718\strut
\end{minipage} & \begin{minipage}[t]{0.07\columnwidth}\raggedleft
0.1906\strut
\end{minipage} & \begin{minipage}[t]{0.07\columnwidth}\raggedleft
0.1839\strut
\end{minipage} & \begin{minipage}[t]{0.07\columnwidth}\raggedleft
0.1764\strut
\end{minipage}\tabularnewline
\begin{minipage}[t]{0.27\columnwidth}\raggedright
Meter Constant\strut
\end{minipage} & \begin{minipage}[t]{0.19\columnwidth}\raggedright
Expansion\strut
\end{minipage} & \begin{minipage}[t]{0.07\columnwidth}\raggedleft
0.3608\strut
\end{minipage} & \begin{minipage}[t]{0.07\columnwidth}\raggedleft
1.1129\strut
\end{minipage} & \begin{minipage}[t]{0.07\columnwidth}\raggedleft
0.5931\strut
\end{minipage} & \begin{minipage}[t]{0.07\columnwidth}\raggedleft
2.1247\strut
\end{minipage} & \begin{minipage}[t]{0.07\columnwidth}\raggedleft
0.3659\strut
\end{minipage}\tabularnewline
\begin{minipage}[t]{0.27\columnwidth}\raggedright
Meter Constant\strut
\end{minipage} & \begin{minipage}[t]{0.19\columnwidth}\raggedright
Orifice\strut
\end{minipage} & \begin{minipage}[t]{0.07\columnwidth}\raggedleft
0.0179\strut
\end{minipage} & \begin{minipage}[t]{0.07\columnwidth}\raggedleft
0.0179\strut
\end{minipage} & \begin{minipage}[t]{0.07\columnwidth}\raggedleft
0.0205\strut
\end{minipage} & \begin{minipage}[t]{0.07\columnwidth}\raggedleft
0.0186\strut
\end{minipage} & \begin{minipage}[t]{0.07\columnwidth}\raggedleft
0.0271\strut
\end{minipage}\tabularnewline
\begin{minipage}[t]{0.27\columnwidth}\raggedright
Meter Constant\strut
\end{minipage} & \begin{minipage}[t]{0.19\columnwidth}\raggedright
Mass Flow Meter\strut
\end{minipage} & \begin{minipage}[t]{0.07\columnwidth}\raggedleft
0.0019\strut
\end{minipage} & \begin{minipage}[t]{0.07\columnwidth}\raggedleft
0.002\strut
\end{minipage} & \begin{minipage}[t]{0.07\columnwidth}\raggedleft
0.0022\strut
\end{minipage} & \begin{minipage}[t]{0.07\columnwidth}\raggedleft
0.002\strut
\end{minipage} & \begin{minipage}[t]{0.07\columnwidth}\raggedleft
0.0023\strut
\end{minipage}\tabularnewline
\begin{minipage}[t]{0.27\columnwidth}\raggedright
\_\_\_\_\_\_\_\_\_\_\_\_\_\_\_\_\_\_\_\_\_\_\_\_\strut
\end{minipage} & \begin{minipage}[t]{0.19\columnwidth}\raggedright
\_\_\_\_\_\_\_\_\_\_\_\_\_\_\_\_\_\strut
\end{minipage} & \begin{minipage}[t]{0.07\columnwidth}\raggedleft
\_\_\_\_\_\_\strut
\end{minipage} & \begin{minipage}[t]{0.07\columnwidth}\raggedleft
\_\_\_\_\_\_\_\strut
\end{minipage} & \begin{minipage}[t]{0.07\columnwidth}\raggedleft
\_\_\_\_\_\_\_\strut
\end{minipage} & \begin{minipage}[t]{0.07\columnwidth}\raggedleft
\_\_\_\_\_\_\_\strut
\end{minipage} & \begin{minipage}[t]{0.07\columnwidth}\raggedleft
\_\_\_\_\_\_\_\strut
\end{minipage}\tabularnewline
\bottomrule
\end{longtable}

The flows can be converted to mass flows by multiplication of the flow
rate with the specific mass (density) of the liquid.

\begin{itemize}
\tightlist
\item
  Include sample calculations for discharge and mass flow.
\end{itemize}

\hypertarget{discussion-of-experimental-results}{%
\subsubsection{4. Discussion of Experimental
Results}\label{discussion-of-experimental-results}}

The purpose of this experiment was to analyze the accuracy and
reliability of different flow measurement devices and compare their
performance against theoretical predictions. Each device was calibrated
using a stopwatch and beaker---our reference measurement---which
provided the most direct and fundamental means of determining flow rate.
The devices under evaluation included a mass flow meter (hall effect
counter), a venturi device, an expansion fitting, an orifice plate, and
a rotameter.

\hypertarget{accuracy-and-reliability-of-devices}{%
\paragraph{Accuracy and Reliability of
Devices}\label{accuracy-and-reliability-of-devices}}

Among the devices tested, the mass flow meter, orifice plate, and
rotameter demonstrated the most stable performance, returning consistent
meter constants across multiple trials. This suggests that their
measurement mechanisms are inherently more reliable under our test
conditions. The mass flow meter, in particular, relies on a turbine
rotation count correlated to flow rate through a predefined meter
constant. While the theoretical meter constant was used initially, some
deviation from expected values was observed. Refining this constant
based on experimental data may yield a more accurate calibration.

\begin{longtable}[]{@{}lrr@{}}
\toprule
Device & Meter Constant & Standard Deviation\tabularnewline
\midrule
\endhead
Venturi & 0.1763 & 0.0122\tabularnewline
Expansion & 0.0911 & 0.7441\tabularnewline
Orifice & 0.0204 & 0.0039\tabularnewline
Rotameter & 0.0025 & 0.0006\tabularnewline
Mass Flow Meter & 0.0021 & 0.0001\tabularnewline
\_\_\_\_\_\_\_\_\_\_\_\_\_\_\_\_\_\_\_\_\_\_\_\_ &
\_\_\_\_\_\_\_\_\_\_\_\_\_\_\_\_\_ & \_\_\_\_\_\_\tabularnewline
\bottomrule
\end{longtable}

In contrast, the venturi device and expansion fitting exhibited
significant measurement variability, with standard deviations comparable
to or exceeding the mean constant value by an order of magnitude.

One probable source of error was the use of manometers to gauge pressure
differences. The manometers proved difficult to read with sufficient
precision, and minor fluctuations in fluid levels led to inconsistencies
in derived flow rates. Additionally, the pressure changes measured
across these devices were not large enough to generate a distinct and
meaningful difference between readings, reducing the effectiveness of
these methods for accurate flow measurement.

\hypertarget{comparison-with-theoretical-predictions}{%
\paragraph{Comparison with Theoretical
Predictions}\label{comparison-with-theoretical-predictions}}

Theoretically, each device should follow well-established flow
equations---Bernoulli's equation for differential pressure-based meters
(venturi, expansion, orifice plate), empirical correlations for the
rotameter, and manufacturer specifications for the mass flow meter.
While the mass flow meter and orifice plate remained within expected
tolerances, the venturi and expansion fittings deviated significantly.
The expansion fitting, in particular, demonstrated poor reliability,
reinforcing the idea that it is not a suitable method for precise flow
measurement. Its sensitivity to small pressure changes, compounded by
possible air entrapment or minor obstructions, likely contributed to its
erratic readings.

\hypertarget{sources-of-error-and-their-impact}{%
\paragraph{Sources of Error and Their
Impact}\label{sources-of-error-and-their-impact}}

Several key sources of error influenced our results:

\begin{enumerate}
\def\labelenumi{\arabic{enumi}.}
\tightlist
\item
  Manometer Reading Difficulties -- Small fluid level changes were
  challenging to distinguish, making pressure-based calculations prone
  to error.
\item
  Measurement Resolution -- Some devices did not produce large enough
  differences in readings to be reliably distinguished from background
  variation.
\item
  Calibration and Meter Constants -- The mass flow meter's performance
  was dependent on its meter constant. While generally reliable,
  refining the constant based on experimental results could improve
  accuracy.
\item
  Flow Variability -- Minor pulsations or fluctuations in flow rate
  could have introduced additional inconsistencies, particularly in
  devices sensitive to small changes in pressure.
\end{enumerate}

Overall, while the mass flow meter, orifice plate, and rotameter
provided useful and repeatable data, the venturi device and expansion
fitting were less effective in this experimental setup. The expansion
fitting, in particular, should not be relied upon for accurate flow
measurement due to its inherent instability and sensitivity to
measurement errors.

    \hypertarget{conclusion}{%
\subsubsection{5. Conclusion}\label{conclusion}}

This experiment evaluated the accuracy and reliability of various flow
measurement devices, with results highlighting both effective and
ineffective methods under the given conditions. The key findings are
summarized as follows:

\begin{itemize}
\item
  Reliable Devices: The mass flow meter, orifice plate, and rotameter
  produced consistent meter constants, indicating that they can be
  reliably used for flow measurement when properly calibrated. The mass
  flow meter, in particular, exhibited a very small standard deviation,
  making it a strong candidate for precise liquid flow measurements,
  provided the liquid is compatible with its operating materials.
\item
  Unreliable Devices: The venturi device and expansion fitting showed
  significant variability, with standard deviations on the same order as
  or exceeding their mean meter constants. This suggests that these
  devices, under the tested conditions, were ineffective for accurate
  flow measurement. The primary sources of error were difficulties in
  reading manometers and insufficient pressure differentials to yield
  meaningful readings.
\item
  Experimental Validation of Flow Equations: The experimentally
  determined meter constants will be presented alongside the governing
  equations for each instrument, demonstrating how theoretical
  principles apply in practice and where deviations occur. The mass flow
  meter and orifice plate aligned well with theoretical expectations,
  while the venturi and expansion devices deviated significantly due to
  measurement limitations.
\end{itemize}

\hypertarget{rotameter}{%
\paragraph{Rotameter}\label{rotameter}}

The equation for using the Rotameter meter is

\$Q = C\_\{rotameter\} \textasciitilde{}\text{Plummet Location} \$

with

\(C_{rotameter} = 0.0025\)

\textbf{Suitable for measurements in future laboratory experiments}
because it is accurate, but not moveable to other lab set ups.

\begin{center}\rule{0.5\linewidth}{0.5pt}\end{center}

\hypertarget{orifice-plate}{%
\paragraph{Orifice Plate}\label{orifice-plate}}

The equation for using the orifice plate as a flow meter is

\$Q = C\_\{orifice\}
A\_u\textsubscript{(\frac{2g}{1-\frac{A_d^2}{A_u^2}})}\sqrt{\frac{P_u}{\rho g}-\frac{P_d}{\rho g}}
\$

with

\(C_{orifice} = 0.0204\)

\hypertarget{expansion-meter}{%
\paragraph{Expansion Meter}\label{expansion-meter}}

The equation for using an expansion as a meter is

\$Q = C\_\{expansion\}
\frac{\sqrt{2g(h_1-h_2)}}{\frac{1}{A_1}-\frac{1}{A_2}} \$

with

\(C_{expansion} = 0.0911\)

\textbf{Not recomended for vital measurements in future laboratory
experiments}

\begin{center}\rule{0.5\linewidth}{0.5pt}\end{center}

\hypertarget{implications-for-fluid-mechanics}{%
\paragraph{Implications for Fluid
Mechanics}\label{implications-for-fluid-mechanics}}

These findings reinforce the importance of selecting appropriate flow
measurement methods based on precision requirements and operating
conditions. Devices that rely on pressure differentials must generate
sufficiently large and readable values to be effective, while those with
inherently stable sensor-based mechanisms, such as the mass flow meter,
can provide highly reliable data. The study highlights the necessity of
experimental validation in fluid mechanics applications, ensuring that
theoretical assumptions align with real-world performance.

    \hypertarget{references}{%
\subsubsection{6. References}\label{references}}

\begin{enumerate}
\def\labelenumi{\arabic{enumi}.}
\tightlist
\item
  Holman, J.P., Experimental Methods for Engineers, 8th Ed.,
  McGraw-Hill, 2012.
\item
  \href{http://54.243.252.9/ce-3105-webroot/2-Exercises/laboratory1/EFM-2.pdf}{Engineering
  Fluid Mechanics - Chapter 2}
\item
  \href{http://54.243.252.9/ce-3105-webroot/2-Exercises/laboratory1/EFM-22.pdf}{Engineering
  Fluid Mechanics - Chapter 11 (Cd for spheres, p.414)}
\item
  \href{https://mech.at.ua/HolmanICS.pdf}{Holman, J.P. (2012)
  \emph{Experimental Methods for Engineers}, 8th Ed. (Chapters 1-3)}
\item
  \href{http://54.243.252.9/ce-3105-webroot/ce3105notes/_build/html/lessons/laboratory3/massflowmeters.html}{Mass
  Flow Meter \& Datalogger Scripts} Programs used in the laboratory
\end{enumerate}

    \hypertarget{analysis-scripts}{%
\subsubsection{7. Analysis Scripts}\label{analysis-scripts}}

    \begin{tcolorbox}[breakable, size=fbox, boxrule=1pt, pad at break*=1mm,colback=cellbackground, colframe=cellborder]
\prompt{In}{incolor}{83}{\boxspacing}
\begin{Verbatim}[commandchars=\\\{\}]
\PY{k}{def} \PY{n+nf}{beakerflow}\PY{p}{(}\PY{n}{volume}\PY{p}{,}\PY{n}{time2fill}\PY{p}{)}\PY{p}{:}
    \PY{n}{beakerflow} \PY{o}{=} \PY{n}{volume}\PY{o}{/}\PY{n}{time2fill}
    \PY{k}{return} \PY{n}{beakerflow}

\PY{n}{volume} \PY{o}{=} \PY{l+m+mi}{2} \PY{c+c1}{\PYZsh{}liters}
\PY{n}{time2fill} \PY{o}{=} \PY{l+m+mf}{27.55} \PY{c+c1}{\PYZsh{}seconds}

\PY{n+nb}{print}\PY{p}{(}\PY{l+s+s2}{\PYZdq{}}\PY{l+s+s2}{Flow rate by beaker time to fill = }\PY{l+s+s2}{\PYZdq{}}\PY{p}{,}\PY{n+nb}{round}\PY{p}{(}\PY{n}{beakerflow}\PY{p}{(}\PY{n}{volume}\PY{p}{,}\PY{n}{time2fill}\PY{p}{)}\PY{p}{,}\PY{l+m+mi}{4}\PY{p}{)}\PY{p}{,}\PY{l+s+s2}{\PYZdq{}}\PY{l+s+s2}{ liters/second}\PY{l+s+s2}{\PYZdq{}}\PY{p}{)}
\end{Verbatim}
\end{tcolorbox}

    \begin{Verbatim}[commandchars=\\\{\}]
Flow rate by beaker time to fill =  0.0726  liters/second
    \end{Verbatim}

    \begin{tcolorbox}[breakable, size=fbox, boxrule=1pt, pad at break*=1mm,colback=cellbackground, colframe=cellborder]
\prompt{In}{incolor}{84}{\boxspacing}
\begin{Verbatim}[commandchars=\\\{\}]
\PY{k}{def} \PY{n+nf}{meterconstant}\PY{p}{(}\PY{n}{flowrate}\PY{p}{,}\PY{n}{counts}\PY{p}{)}\PY{p}{:}
    \PY{n}{meterconstant} \PY{o}{=} \PY{n}{flowrate}\PY{o}{/}\PY{n+nb}{float}\PY{p}{(}\PY{n}{counts}\PY{p}{)}
    \PY{k}{return} \PY{n}{meterconstant}

\PY{n}{flowrate} \PY{o}{=} \PY{l+m+mf}{0.0726} \PY{c+c1}{\PYZsh{}LPS}
\PY{n}{counts} \PY{o}{=} \PY{l+m+mi}{31} \PY{c+c1}{\PYZsh{}counts/dwell time}

\PY{n+nb}{print}\PY{p}{(}\PY{l+s+s2}{\PYZdq{}}\PY{l+s+s2}{Meter Constant = }\PY{l+s+s2}{\PYZdq{}}\PY{p}{,}\PY{n+nb}{round}\PY{p}{(}\PY{n}{meterconstant}\PY{p}{(}\PY{n}{flowrate}\PY{p}{,}\PY{n}{counts}\PY{p}{)}\PY{p}{,}\PY{l+m+mi}{4}\PY{p}{)}\PY{p}{,}\PY{l+s+s2}{\PYZdq{}}\PY{l+s+s2}{ LPS/count}\PY{l+s+s2}{\PYZdq{}}\PY{p}{)}
\end{Verbatim}
\end{tcolorbox}

    \begin{Verbatim}[commandchars=\\\{\}]
Meter Constant =  0.0023  LPS/count
    \end{Verbatim}

    \begin{tcolorbox}[breakable, size=fbox, boxrule=1pt, pad at break*=1mm,colback=cellbackground, colframe=cellborder]
\prompt{In}{incolor}{85}{\boxspacing}
\begin{Verbatim}[commandchars=\\\{\}]
\PY{k}{def} \PY{n+nf}{venturi}\PY{p}{(}\PY{n}{cd}\PY{p}{,}\PY{n}{a1}\PY{p}{,}\PY{n}{a2}\PY{p}{,}\PY{n}{h1}\PY{p}{,}\PY{n}{h2}\PY{p}{,}\PY{n}{g}\PY{p}{)}\PY{p}{:}
    \PY{k+kn}{import} \PY{n+nn}{math}
    \PY{n}{venturi} \PY{o}{=} \PY{n}{cd}\PY{o}{*}\PY{n}{a2}\PY{o}{*}\PY{p}{(}\PY{p}{(}\PY{l+m+mi}{2}\PY{o}{*}\PY{n}{g}\PY{p}{)}\PY{o}{/}\PY{p}{(}\PY{l+m+mi}{1}\PY{o}{\PYZhy{}}\PY{p}{(}\PY{n}{a2}\PY{o}{*}\PY{o}{*}\PY{l+m+mi}{2}\PY{p}{)}\PY{o}{/}\PY{p}{(}\PY{n}{a1}\PY{o}{*}\PY{o}{*}\PY{l+m+mi}{2}\PY{p}{)}\PY{p}{)}\PY{p}{)}\PY{o}{*}\PY{n}{math}\PY{o}{.}\PY{n}{sqrt}\PY{p}{(}\PY{n}{h1}\PY{o}{\PYZhy{}}\PY{n}{h2}\PY{p}{)}
    \PY{k}{return} \PY{n}{venturi}

\PY{n}{cventuri} \PY{o}{=} \PY{l+m+mf}{1.0}
\PY{n}{area1} \PY{o}{=} \PY{l+m+mf}{530.9} \PY{c+c1}{\PYZsh{}mm\PYZca{}2}
\PY{n}{area2} \PY{o}{=} \PY{l+m+mf}{201.1} \PY{c+c1}{\PYZsh{}mm\PYZca{}2}
\PY{n}{h1} \PY{o}{=} \PY{l+m+mi}{158} \PY{c+c1}{\PYZsh{}mm}
\PY{n}{h2} \PY{o}{=} \PY{l+m+mi}{110} \PY{c+c1}{\PYZsh{}mm}
\PY{n}{g} \PY{o}{=} \PY{l+m+mf}{9.800} \PY{c+c1}{\PYZsh{}m/s\PYZca{}2}
\PY{n}{area1} \PY{o}{=} \PY{n}{area1}\PY{o}{/}\PY{l+m+mi}{1000}\PY{o}{/}\PY{l+m+mi}{1000}
\PY{n}{area2} \PY{o}{=} \PY{n}{area2}\PY{o}{/}\PY{l+m+mi}{1000}\PY{o}{/}\PY{l+m+mi}{1000}
\PY{n}{h1} \PY{o}{=} \PY{n}{h1}\PY{o}{/}\PY{l+m+mi}{1000}
\PY{n}{h2} \PY{o}{=} \PY{n}{h2}\PY{o}{/}\PY{l+m+mi}{1000}

\PY{n+nb}{print}\PY{p}{(}\PY{l+s+s2}{\PYZdq{}}\PY{l+s+s2}{Venturi Flow Rate }\PY{l+s+s2}{\PYZdq{}}\PY{p}{,}\PY{n+nb}{round}\PY{p}{(}\PY{l+m+mi}{1000}\PY{o}{*}\PY{n}{venturi}\PY{p}{(}\PY{n}{cventuri}\PY{p}{,}\PY{n}{area1}\PY{p}{,}\PY{n}{area2}\PY{p}{,}\PY{n}{h1}\PY{p}{,}\PY{n}{h2}\PY{p}{,}\PY{n}{g}\PY{p}{)}\PY{p}{,}\PY{l+m+mi}{4}\PY{p}{)}\PY{p}{,}\PY{l+s+s2}{\PYZdq{}}\PY{l+s+s2}{ LPS }\PY{l+s+s2}{\PYZdq{}}\PY{p}{)}
\end{Verbatim}
\end{tcolorbox}

    \begin{Verbatim}[commandchars=\\\{\}]
Venturi Flow Rate  1.0082  LPS
    \end{Verbatim}

    \begin{tcolorbox}[breakable, size=fbox, boxrule=1pt, pad at break*=1mm,colback=cellbackground, colframe=cellborder]
\prompt{In}{incolor}{86}{\boxspacing}
\begin{Verbatim}[commandchars=\\\{\}]
\PY{k}{def} \PY{n+nf}{orifice}\PY{p}{(}\PY{n}{cd}\PY{p}{,}\PY{n}{a1}\PY{p}{,}\PY{n}{a2}\PY{p}{,}\PY{n}{h1}\PY{p}{,}\PY{n}{h2}\PY{p}{,}\PY{n}{g}\PY{p}{)}\PY{p}{:}
    \PY{k+kn}{import} \PY{n+nn}{math}
    \PY{n}{orifice} \PY{o}{=} \PY{n}{cd}\PY{o}{*}\PY{n}{a1}\PY{o}{*}\PY{p}{(}\PY{p}{(}\PY{l+m+mi}{2}\PY{o}{*}\PY{n}{g}\PY{p}{)}\PY{o}{/}\PY{p}{(}\PY{l+m+mi}{1}\PY{o}{\PYZhy{}}\PY{p}{(}\PY{n}{a2}\PY{o}{*}\PY{o}{*}\PY{l+m+mi}{2}\PY{p}{)}\PY{o}{/}\PY{p}{(}\PY{n}{a1}\PY{o}{*}\PY{o}{*}\PY{l+m+mi}{2}\PY{p}{)}\PY{p}{)}\PY{p}{)}\PY{o}{*}\PY{n}{math}\PY{o}{.}\PY{n}{sqrt}\PY{p}{(}\PY{n}{h1}\PY{o}{\PYZhy{}}\PY{n}{h2}\PY{p}{)}
    \PY{k}{return} \PY{n}{orifice}

\PY{n}{corifice} \PY{o}{=} \PY{l+m+mf}{1.0}
\PY{n}{area1} \PY{o}{=} \PY{l+m+mi}{2116} \PY{c+c1}{\PYZsh{}mm\PYZca{}2}
\PY{n}{area2} \PY{o}{=} \PY{l+m+mf}{314.16} \PY{c+c1}{\PYZsh{}mm\PYZca{}2}
\PY{n}{h1} \PY{o}{=} \PY{l+m+mi}{306} \PY{c+c1}{\PYZsh{}mm}
\PY{n}{h2} \PY{o}{=} \PY{l+m+mi}{302} \PY{c+c1}{\PYZsh{}mm}
\PY{n}{g} \PY{o}{=} \PY{l+m+mf}{9.800} \PY{c+c1}{\PYZsh{}m/s\PYZca{}2}
\PY{n}{area1} \PY{o}{=} \PY{n}{area1}\PY{o}{/}\PY{l+m+mi}{1000}\PY{o}{/}\PY{l+m+mi}{1000}
\PY{n}{area2} \PY{o}{=} \PY{n}{area2}\PY{o}{/}\PY{l+m+mi}{1000}\PY{o}{/}\PY{l+m+mi}{1000}
\PY{n}{h1} \PY{o}{=} \PY{n}{h1}\PY{o}{/}\PY{l+m+mi}{1000}
\PY{n}{h2} \PY{o}{=} \PY{n}{h2}\PY{o}{/}\PY{l+m+mi}{1000}

\PY{n+nb}{print}\PY{p}{(}\PY{l+s+s2}{\PYZdq{}}\PY{l+s+s2}{Orifice Flow Rate }\PY{l+s+s2}{\PYZdq{}}\PY{p}{,}\PY{n+nb}{round}\PY{p}{(}\PY{l+m+mi}{1000}\PY{o}{*}\PY{n}{orifice}\PY{p}{(}\PY{n}{corifice}\PY{p}{,}\PY{n}{area1}\PY{p}{,}\PY{n}{area2}\PY{p}{,}\PY{n}{h1}\PY{p}{,}\PY{n}{h2}\PY{p}{,}\PY{n}{g}\PY{p}{)}\PY{p}{,}\PY{l+m+mi}{4}\PY{p}{)}\PY{p}{,}\PY{l+s+s2}{\PYZdq{}}\PY{l+s+s2}{ LPS }\PY{l+s+s2}{\PYZdq{}}\PY{p}{)}
\end{Verbatim}
\end{tcolorbox}

    \begin{Verbatim}[commandchars=\\\{\}]
Orifice Flow Rate  2.6821  LPS
    \end{Verbatim}

    \begin{tcolorbox}[breakable, size=fbox, boxrule=1pt, pad at break*=1mm,colback=cellbackground, colframe=cellborder]
\prompt{In}{incolor}{87}{\boxspacing}
\begin{Verbatim}[commandchars=\\\{\}]
\PY{k}{def} \PY{n+nf}{expansion}\PY{p}{(}\PY{n}{cd}\PY{p}{,}\PY{n}{a1}\PY{p}{,}\PY{n}{a2}\PY{p}{,}\PY{n}{h1}\PY{p}{,}\PY{n}{h2}\PY{p}{,}\PY{n}{g}\PY{p}{)}\PY{p}{:}
    \PY{k+kn}{import} \PY{n+nn}{math}
    \PY{n}{expansion} \PY{o}{=} \PY{n}{cd}\PY{o}{*}\PY{n}{math}\PY{o}{.}\PY{n}{sqrt}\PY{p}{(}\PY{l+m+mi}{2}\PY{o}{*}\PY{n}{g}\PY{o}{*}\PY{n+nb}{abs}\PY{p}{(}\PY{n}{h1}\PY{o}{\PYZhy{}}\PY{n}{h2}\PY{p}{)}\PY{p}{)}\PY{o}{/}\PY{p}{(}\PY{p}{(}\PY{l+m+mi}{1}\PY{o}{/}\PY{n}{a1}\PY{p}{)}\PY{o}{\PYZhy{}}\PY{p}{(}\PY{l+m+mi}{1}\PY{o}{/}\PY{n}{a2}\PY{p}{)}\PY{p}{)}
    \PY{k}{return} \PY{n}{expansion}

\PY{n}{cexpansion} \PY{o}{=} \PY{l+m+mf}{1.0}
\PY{n}{area1} \PY{o}{=} \PY{l+m+mf}{530.9} \PY{c+c1}{\PYZsh{}mm\PYZca{}2}
\PY{n}{area2} \PY{o}{=} \PY{l+m+mi}{2116} \PY{c+c1}{\PYZsh{}mm\PYZca{}2}
\PY{n}{h1} \PY{o}{=} \PY{l+m+mi}{306} \PY{c+c1}{\PYZsh{}mm}
\PY{n}{h2} \PY{o}{=} \PY{l+m+mi}{302} \PY{c+c1}{\PYZsh{}mm}
\PY{n}{g} \PY{o}{=} \PY{l+m+mf}{9.800} \PY{c+c1}{\PYZsh{}m/s\PYZca{}2}
\PY{n}{area1} \PY{o}{=} \PY{n}{area1}\PY{o}{/}\PY{l+m+mi}{1000}\PY{o}{/}\PY{l+m+mi}{1000}
\PY{n}{area2} \PY{o}{=} \PY{n}{area2}\PY{o}{/}\PY{l+m+mi}{1000}\PY{o}{/}\PY{l+m+mi}{1000}
\PY{n}{h1} \PY{o}{=} \PY{n}{h1}\PY{o}{/}\PY{l+m+mi}{1000}
\PY{n}{h2} \PY{o}{=} \PY{n}{h2}\PY{o}{/}\PY{l+m+mi}{1000}

\PY{n+nb}{print}\PY{p}{(}\PY{l+s+s2}{\PYZdq{}}\PY{l+s+s2}{Expansion Flow Rate }\PY{l+s+s2}{\PYZdq{}}\PY{p}{,}\PY{n+nb}{round}\PY{p}{(}\PY{l+m+mi}{1000}\PY{o}{*}\PY{n}{expansion}\PY{p}{(}\PY{n}{cexpansion}\PY{p}{,}\PY{n}{area1}\PY{p}{,}\PY{n}{area2}\PY{p}{,}\PY{n}{h1}\PY{p}{,}\PY{n}{h2}\PY{p}{,}\PY{n}{g}\PY{p}{)}\PY{p}{,}\PY{l+m+mi}{4}\PY{p}{)}\PY{p}{,}\PY{l+s+s2}{\PYZdq{}}\PY{l+s+s2}{ LPS }\PY{l+s+s2}{\PYZdq{}}\PY{p}{)}
\end{Verbatim}
\end{tcolorbox}

    \begin{Verbatim}[commandchars=\\\{\}]
Expansion Flow Rate  0.1984  LPS
    \end{Verbatim}

    \begin{tcolorbox}[breakable, size=fbox, boxrule=1pt, pad at break*=1mm,colback=cellbackground, colframe=cellborder]
\prompt{In}{incolor}{88}{\boxspacing}
\begin{Verbatim}[commandchars=\\\{\}]
\PY{k+kn}{import} \PY{n+nn}{math}
\PY{n}{d} \PY{o}{=} \PY{l+m+mi}{20}\PY{o}{/}\PY{l+m+mi}{1000}
\PY{n}{area} \PY{o}{=} \PY{l+m+mf}{0.25}\PY{o}{*}\PY{n}{math}\PY{o}{.}\PY{n}{pi}\PY{o}{*}\PY{n}{d}\PY{o}{*}\PY{o}{*}\PY{l+m+mi}{2}
\PY{n+nb}{print}\PY{p}{(}\PY{l+s+s2}{\PYZdq{}}\PY{l+s+s2}{Cross Section Area = }\PY{l+s+s2}{\PYZdq{}}\PY{p}{,}\PY{n+nb}{round}\PY{p}{(}\PY{n}{area}\PY{o}{*}\PY{l+m+mi}{1000}\PY{o}{*}\PY{l+m+mi}{1000}\PY{p}{,}\PY{l+m+mi}{3}\PY{p}{)}\PY{p}{,}\PY{l+s+s2}{\PYZdq{}}\PY{l+s+s2}{ sq.mm. }\PY{l+s+s2}{\PYZdq{}}\PY{p}{)}
\end{Verbatim}
\end{tcolorbox}

    \begin{Verbatim}[commandchars=\\\{\}]
Cross Section Area =  314.159  sq.mm.
    \end{Verbatim}


    % Add a bibliography block to the postdoc
    
    
    
\end{document}
