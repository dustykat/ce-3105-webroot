\documentclass[11pt]{article}

    \usepackage[breakable]{tcolorbox}
    \usepackage{parskip} % Stop auto-indenting (to mimic markdown behaviour)
    

    % Basic figure setup, for now with no caption control since it's done
    % automatically by Pandoc (which extracts ![](path) syntax from Markdown).
    \usepackage{graphicx}
    % Keep aspect ratio if custom image width or height is specified
    \setkeys{Gin}{keepaspectratio}
    % Maintain compatibility with old templates. Remove in nbconvert 6.0
    \let\Oldincludegraphics\includegraphics
    % Ensure that by default, figures have no caption (until we provide a
    % proper Figure object with a Caption API and a way to capture that
    % in the conversion process - todo).
    \usepackage{caption}
    \DeclareCaptionFormat{nocaption}{}
    \captionsetup{format=nocaption,aboveskip=0pt,belowskip=0pt}

    \usepackage{float}
    \floatplacement{figure}{H} % forces figures to be placed at the correct location
    \usepackage{xcolor} % Allow colors to be defined
    \usepackage{enumerate} % Needed for markdown enumerations to work
    \usepackage{geometry} % Used to adjust the document margins
    \usepackage{amsmath} % Equations
    \usepackage{amssymb} % Equations
    \usepackage{textcomp} % defines textquotesingle
    % Hack from http://tex.stackexchange.com/a/47451/13684:
    \AtBeginDocument{%
        \def\PYZsq{\textquotesingle}% Upright quotes in Pygmentized code
    }
    \usepackage{upquote} % Upright quotes for verbatim code
    \usepackage{eurosym} % defines \euro

    \usepackage{iftex}
    \ifPDFTeX
        \usepackage[T1]{fontenc}
        \IfFileExists{alphabeta.sty}{
              \usepackage{alphabeta}
          }{
              \usepackage[mathletters]{ucs}
              \usepackage[utf8x]{inputenc}
          }
    \else
        \usepackage{fontspec}
        \usepackage{unicode-math}
    \fi

    \usepackage{fancyvrb} % verbatim replacement that allows latex
    \usepackage{grffile} % extends the file name processing of package graphics
                         % to support a larger range
    \makeatletter % fix for old versions of grffile with XeLaTeX
    \@ifpackagelater{grffile}{2019/11/01}
    {
      % Do nothing on new versions
    }
    {
      \def\Gread@@xetex#1{%
        \IfFileExists{"\Gin@base".bb}%
        {\Gread@eps{\Gin@base.bb}}%
        {\Gread@@xetex@aux#1}%
      }
    }
    \makeatother
    \usepackage[Export]{adjustbox} % Used to constrain images to a maximum size
    \adjustboxset{max size={0.9\linewidth}{0.9\paperheight}}

    % The hyperref package gives us a pdf with properly built
    % internal navigation ('pdf bookmarks' for the table of contents,
    % internal cross-reference links, web links for URLs, etc.)
    \usepackage{hyperref}
    % The default LaTeX title has an obnoxious amount of whitespace. By default,
    % titling removes some of it. It also provides customization options.
    \usepackage{titling}
    \usepackage{longtable} % longtable support required by pandoc >1.10
    \usepackage{booktabs}  % table support for pandoc > 1.12.2
    \usepackage{array}     % table support for pandoc >= 2.11.3
    \usepackage{calc}      % table minipage width calculation for pandoc >= 2.11.1
    \usepackage[inline]{enumitem} % IRkernel/repr support (it uses the enumerate* environment)
    \usepackage[normalem]{ulem} % ulem is needed to support strikethroughs (\sout)
                                % normalem makes italics be italics, not underlines
    \usepackage{soul}      % strikethrough (\st) support for pandoc >= 3.0.0
    \usepackage{mathrsfs}
    

    
    % Colors for the hyperref package
    \definecolor{urlcolor}{rgb}{0,.145,.698}
    \definecolor{linkcolor}{rgb}{.71,0.21,0.01}
    \definecolor{citecolor}{rgb}{.12,.54,.11}

    % ANSI colors
    \definecolor{ansi-black}{HTML}{3E424D}
    \definecolor{ansi-black-intense}{HTML}{282C36}
    \definecolor{ansi-red}{HTML}{E75C58}
    \definecolor{ansi-red-intense}{HTML}{B22B31}
    \definecolor{ansi-green}{HTML}{00A250}
    \definecolor{ansi-green-intense}{HTML}{007427}
    \definecolor{ansi-yellow}{HTML}{DDB62B}
    \definecolor{ansi-yellow-intense}{HTML}{B27D12}
    \definecolor{ansi-blue}{HTML}{208FFB}
    \definecolor{ansi-blue-intense}{HTML}{0065CA}
    \definecolor{ansi-magenta}{HTML}{D160C4}
    \definecolor{ansi-magenta-intense}{HTML}{A03196}
    \definecolor{ansi-cyan}{HTML}{60C6C8}
    \definecolor{ansi-cyan-intense}{HTML}{258F8F}
    \definecolor{ansi-white}{HTML}{C5C1B4}
    \definecolor{ansi-white-intense}{HTML}{A1A6B2}
    \definecolor{ansi-default-inverse-fg}{HTML}{FFFFFF}
    \definecolor{ansi-default-inverse-bg}{HTML}{000000}

    % common color for the border for error outputs.
    \definecolor{outerrorbackground}{HTML}{FFDFDF}

    % commands and environments needed by pandoc snippets
    % extracted from the output of `pandoc -s`
    \providecommand{\tightlist}{%
      \setlength{\itemsep}{0pt}\setlength{\parskip}{0pt}}
    \DefineVerbatimEnvironment{Highlighting}{Verbatim}{commandchars=\\\{\}}
    % Add ',fontsize=\small' for more characters per line
    \newenvironment{Shaded}{}{}
    \newcommand{\KeywordTok}[1]{\textcolor[rgb]{0.00,0.44,0.13}{\textbf{{#1}}}}
    \newcommand{\DataTypeTok}[1]{\textcolor[rgb]{0.56,0.13,0.00}{{#1}}}
    \newcommand{\DecValTok}[1]{\textcolor[rgb]{0.25,0.63,0.44}{{#1}}}
    \newcommand{\BaseNTok}[1]{\textcolor[rgb]{0.25,0.63,0.44}{{#1}}}
    \newcommand{\FloatTok}[1]{\textcolor[rgb]{0.25,0.63,0.44}{{#1}}}
    \newcommand{\CharTok}[1]{\textcolor[rgb]{0.25,0.44,0.63}{{#1}}}
    \newcommand{\StringTok}[1]{\textcolor[rgb]{0.25,0.44,0.63}{{#1}}}
    \newcommand{\CommentTok}[1]{\textcolor[rgb]{0.38,0.63,0.69}{\textit{{#1}}}}
    \newcommand{\OtherTok}[1]{\textcolor[rgb]{0.00,0.44,0.13}{{#1}}}
    \newcommand{\AlertTok}[1]{\textcolor[rgb]{1.00,0.00,0.00}{\textbf{{#1}}}}
    \newcommand{\FunctionTok}[1]{\textcolor[rgb]{0.02,0.16,0.49}{{#1}}}
    \newcommand{\RegionMarkerTok}[1]{{#1}}
    \newcommand{\ErrorTok}[1]{\textcolor[rgb]{1.00,0.00,0.00}{\textbf{{#1}}}}
    \newcommand{\NormalTok}[1]{{#1}}

    % Additional commands for more recent versions of Pandoc
    \newcommand{\ConstantTok}[1]{\textcolor[rgb]{0.53,0.00,0.00}{{#1}}}
    \newcommand{\SpecialCharTok}[1]{\textcolor[rgb]{0.25,0.44,0.63}{{#1}}}
    \newcommand{\VerbatimStringTok}[1]{\textcolor[rgb]{0.25,0.44,0.63}{{#1}}}
    \newcommand{\SpecialStringTok}[1]{\textcolor[rgb]{0.73,0.40,0.53}{{#1}}}
    \newcommand{\ImportTok}[1]{{#1}}
    \newcommand{\DocumentationTok}[1]{\textcolor[rgb]{0.73,0.13,0.13}{\textit{{#1}}}}
    \newcommand{\AnnotationTok}[1]{\textcolor[rgb]{0.38,0.63,0.69}{\textbf{\textit{{#1}}}}}
    \newcommand{\CommentVarTok}[1]{\textcolor[rgb]{0.38,0.63,0.69}{\textbf{\textit{{#1}}}}}
    \newcommand{\VariableTok}[1]{\textcolor[rgb]{0.10,0.09,0.49}{{#1}}}
    \newcommand{\ControlFlowTok}[1]{\textcolor[rgb]{0.00,0.44,0.13}{\textbf{{#1}}}}
    \newcommand{\OperatorTok}[1]{\textcolor[rgb]{0.40,0.40,0.40}{{#1}}}
    \newcommand{\BuiltInTok}[1]{{#1}}
    \newcommand{\ExtensionTok}[1]{{#1}}
    \newcommand{\PreprocessorTok}[1]{\textcolor[rgb]{0.74,0.48,0.00}{{#1}}}
    \newcommand{\AttributeTok}[1]{\textcolor[rgb]{0.49,0.56,0.16}{{#1}}}
    \newcommand{\InformationTok}[1]{\textcolor[rgb]{0.38,0.63,0.69}{\textbf{\textit{{#1}}}}}
    \newcommand{\WarningTok}[1]{\textcolor[rgb]{0.38,0.63,0.69}{\textbf{\textit{{#1}}}}}


    % Define a nice break command that doesn't care if a line doesn't already
    % exist.
    \def\br{\hspace*{\fill} \\* }
    % Math Jax compatibility definitions
    \def\gt{>}
    \def\lt{<}
    \let\Oldtex\TeX
    \let\Oldlatex\LaTeX
    \renewcommand{\TeX}{\textrm{\Oldtex}}
    \renewcommand{\LaTeX}{\textrm{\Oldlatex}}
    % Document parameters
    % Document title
    \title{laboratory1}
    
    
    
    
    
    
    
% Pygments definitions
\makeatletter
\def\PY@reset{\let\PY@it=\relax \let\PY@bf=\relax%
    \let\PY@ul=\relax \let\PY@tc=\relax%
    \let\PY@bc=\relax \let\PY@ff=\relax}
\def\PY@tok#1{\csname PY@tok@#1\endcsname}
\def\PY@toks#1+{\ifx\relax#1\empty\else%
    \PY@tok{#1}\expandafter\PY@toks\fi}
\def\PY@do#1{\PY@bc{\PY@tc{\PY@ul{%
    \PY@it{\PY@bf{\PY@ff{#1}}}}}}}
\def\PY#1#2{\PY@reset\PY@toks#1+\relax+\PY@do{#2}}

\@namedef{PY@tok@w}{\def\PY@tc##1{\textcolor[rgb]{0.73,0.73,0.73}{##1}}}
\@namedef{PY@tok@c}{\let\PY@it=\textit\def\PY@tc##1{\textcolor[rgb]{0.24,0.48,0.48}{##1}}}
\@namedef{PY@tok@cp}{\def\PY@tc##1{\textcolor[rgb]{0.61,0.40,0.00}{##1}}}
\@namedef{PY@tok@k}{\let\PY@bf=\textbf\def\PY@tc##1{\textcolor[rgb]{0.00,0.50,0.00}{##1}}}
\@namedef{PY@tok@kp}{\def\PY@tc##1{\textcolor[rgb]{0.00,0.50,0.00}{##1}}}
\@namedef{PY@tok@kt}{\def\PY@tc##1{\textcolor[rgb]{0.69,0.00,0.25}{##1}}}
\@namedef{PY@tok@o}{\def\PY@tc##1{\textcolor[rgb]{0.40,0.40,0.40}{##1}}}
\@namedef{PY@tok@ow}{\let\PY@bf=\textbf\def\PY@tc##1{\textcolor[rgb]{0.67,0.13,1.00}{##1}}}
\@namedef{PY@tok@nb}{\def\PY@tc##1{\textcolor[rgb]{0.00,0.50,0.00}{##1}}}
\@namedef{PY@tok@nf}{\def\PY@tc##1{\textcolor[rgb]{0.00,0.00,1.00}{##1}}}
\@namedef{PY@tok@nc}{\let\PY@bf=\textbf\def\PY@tc##1{\textcolor[rgb]{0.00,0.00,1.00}{##1}}}
\@namedef{PY@tok@nn}{\let\PY@bf=\textbf\def\PY@tc##1{\textcolor[rgb]{0.00,0.00,1.00}{##1}}}
\@namedef{PY@tok@ne}{\let\PY@bf=\textbf\def\PY@tc##1{\textcolor[rgb]{0.80,0.25,0.22}{##1}}}
\@namedef{PY@tok@nv}{\def\PY@tc##1{\textcolor[rgb]{0.10,0.09,0.49}{##1}}}
\@namedef{PY@tok@no}{\def\PY@tc##1{\textcolor[rgb]{0.53,0.00,0.00}{##1}}}
\@namedef{PY@tok@nl}{\def\PY@tc##1{\textcolor[rgb]{0.46,0.46,0.00}{##1}}}
\@namedef{PY@tok@ni}{\let\PY@bf=\textbf\def\PY@tc##1{\textcolor[rgb]{0.44,0.44,0.44}{##1}}}
\@namedef{PY@tok@na}{\def\PY@tc##1{\textcolor[rgb]{0.41,0.47,0.13}{##1}}}
\@namedef{PY@tok@nt}{\let\PY@bf=\textbf\def\PY@tc##1{\textcolor[rgb]{0.00,0.50,0.00}{##1}}}
\@namedef{PY@tok@nd}{\def\PY@tc##1{\textcolor[rgb]{0.67,0.13,1.00}{##1}}}
\@namedef{PY@tok@s}{\def\PY@tc##1{\textcolor[rgb]{0.73,0.13,0.13}{##1}}}
\@namedef{PY@tok@sd}{\let\PY@it=\textit\def\PY@tc##1{\textcolor[rgb]{0.73,0.13,0.13}{##1}}}
\@namedef{PY@tok@si}{\let\PY@bf=\textbf\def\PY@tc##1{\textcolor[rgb]{0.64,0.35,0.47}{##1}}}
\@namedef{PY@tok@se}{\let\PY@bf=\textbf\def\PY@tc##1{\textcolor[rgb]{0.67,0.36,0.12}{##1}}}
\@namedef{PY@tok@sr}{\def\PY@tc##1{\textcolor[rgb]{0.64,0.35,0.47}{##1}}}
\@namedef{PY@tok@ss}{\def\PY@tc##1{\textcolor[rgb]{0.10,0.09,0.49}{##1}}}
\@namedef{PY@tok@sx}{\def\PY@tc##1{\textcolor[rgb]{0.00,0.50,0.00}{##1}}}
\@namedef{PY@tok@m}{\def\PY@tc##1{\textcolor[rgb]{0.40,0.40,0.40}{##1}}}
\@namedef{PY@tok@gh}{\let\PY@bf=\textbf\def\PY@tc##1{\textcolor[rgb]{0.00,0.00,0.50}{##1}}}
\@namedef{PY@tok@gu}{\let\PY@bf=\textbf\def\PY@tc##1{\textcolor[rgb]{0.50,0.00,0.50}{##1}}}
\@namedef{PY@tok@gd}{\def\PY@tc##1{\textcolor[rgb]{0.63,0.00,0.00}{##1}}}
\@namedef{PY@tok@gi}{\def\PY@tc##1{\textcolor[rgb]{0.00,0.52,0.00}{##1}}}
\@namedef{PY@tok@gr}{\def\PY@tc##1{\textcolor[rgb]{0.89,0.00,0.00}{##1}}}
\@namedef{PY@tok@ge}{\let\PY@it=\textit}
\@namedef{PY@tok@gs}{\let\PY@bf=\textbf}
\@namedef{PY@tok@ges}{\let\PY@bf=\textbf\let\PY@it=\textit}
\@namedef{PY@tok@gp}{\let\PY@bf=\textbf\def\PY@tc##1{\textcolor[rgb]{0.00,0.00,0.50}{##1}}}
\@namedef{PY@tok@go}{\def\PY@tc##1{\textcolor[rgb]{0.44,0.44,0.44}{##1}}}
\@namedef{PY@tok@gt}{\def\PY@tc##1{\textcolor[rgb]{0.00,0.27,0.87}{##1}}}
\@namedef{PY@tok@err}{\def\PY@bc##1{{\setlength{\fboxsep}{\string -\fboxrule}\fcolorbox[rgb]{1.00,0.00,0.00}{1,1,1}{\strut ##1}}}}
\@namedef{PY@tok@kc}{\let\PY@bf=\textbf\def\PY@tc##1{\textcolor[rgb]{0.00,0.50,0.00}{##1}}}
\@namedef{PY@tok@kd}{\let\PY@bf=\textbf\def\PY@tc##1{\textcolor[rgb]{0.00,0.50,0.00}{##1}}}
\@namedef{PY@tok@kn}{\let\PY@bf=\textbf\def\PY@tc##1{\textcolor[rgb]{0.00,0.50,0.00}{##1}}}
\@namedef{PY@tok@kr}{\let\PY@bf=\textbf\def\PY@tc##1{\textcolor[rgb]{0.00,0.50,0.00}{##1}}}
\@namedef{PY@tok@bp}{\def\PY@tc##1{\textcolor[rgb]{0.00,0.50,0.00}{##1}}}
\@namedef{PY@tok@fm}{\def\PY@tc##1{\textcolor[rgb]{0.00,0.00,1.00}{##1}}}
\@namedef{PY@tok@vc}{\def\PY@tc##1{\textcolor[rgb]{0.10,0.09,0.49}{##1}}}
\@namedef{PY@tok@vg}{\def\PY@tc##1{\textcolor[rgb]{0.10,0.09,0.49}{##1}}}
\@namedef{PY@tok@vi}{\def\PY@tc##1{\textcolor[rgb]{0.10,0.09,0.49}{##1}}}
\@namedef{PY@tok@vm}{\def\PY@tc##1{\textcolor[rgb]{0.10,0.09,0.49}{##1}}}
\@namedef{PY@tok@sa}{\def\PY@tc##1{\textcolor[rgb]{0.73,0.13,0.13}{##1}}}
\@namedef{PY@tok@sb}{\def\PY@tc##1{\textcolor[rgb]{0.73,0.13,0.13}{##1}}}
\@namedef{PY@tok@sc}{\def\PY@tc##1{\textcolor[rgb]{0.73,0.13,0.13}{##1}}}
\@namedef{PY@tok@dl}{\def\PY@tc##1{\textcolor[rgb]{0.73,0.13,0.13}{##1}}}
\@namedef{PY@tok@s2}{\def\PY@tc##1{\textcolor[rgb]{0.73,0.13,0.13}{##1}}}
\@namedef{PY@tok@sh}{\def\PY@tc##1{\textcolor[rgb]{0.73,0.13,0.13}{##1}}}
\@namedef{PY@tok@s1}{\def\PY@tc##1{\textcolor[rgb]{0.73,0.13,0.13}{##1}}}
\@namedef{PY@tok@mb}{\def\PY@tc##1{\textcolor[rgb]{0.40,0.40,0.40}{##1}}}
\@namedef{PY@tok@mf}{\def\PY@tc##1{\textcolor[rgb]{0.40,0.40,0.40}{##1}}}
\@namedef{PY@tok@mh}{\def\PY@tc##1{\textcolor[rgb]{0.40,0.40,0.40}{##1}}}
\@namedef{PY@tok@mi}{\def\PY@tc##1{\textcolor[rgb]{0.40,0.40,0.40}{##1}}}
\@namedef{PY@tok@il}{\def\PY@tc##1{\textcolor[rgb]{0.40,0.40,0.40}{##1}}}
\@namedef{PY@tok@mo}{\def\PY@tc##1{\textcolor[rgb]{0.40,0.40,0.40}{##1}}}
\@namedef{PY@tok@ch}{\let\PY@it=\textit\def\PY@tc##1{\textcolor[rgb]{0.24,0.48,0.48}{##1}}}
\@namedef{PY@tok@cm}{\let\PY@it=\textit\def\PY@tc##1{\textcolor[rgb]{0.24,0.48,0.48}{##1}}}
\@namedef{PY@tok@cpf}{\let\PY@it=\textit\def\PY@tc##1{\textcolor[rgb]{0.24,0.48,0.48}{##1}}}
\@namedef{PY@tok@c1}{\let\PY@it=\textit\def\PY@tc##1{\textcolor[rgb]{0.24,0.48,0.48}{##1}}}
\@namedef{PY@tok@cs}{\let\PY@it=\textit\def\PY@tc##1{\textcolor[rgb]{0.24,0.48,0.48}{##1}}}

\def\PYZbs{\char`\\}
\def\PYZus{\char`\_}
\def\PYZob{\char`\{}
\def\PYZcb{\char`\}}
\def\PYZca{\char`\^}
\def\PYZam{\char`\&}
\def\PYZlt{\char`\<}
\def\PYZgt{\char`\>}
\def\PYZsh{\char`\#}
\def\PYZpc{\char`\%}
\def\PYZdl{\char`\$}
\def\PYZhy{\char`\-}
\def\PYZsq{\char`\'}
\def\PYZdq{\char`\"}
\def\PYZti{\char`\~}
% for compatibility with earlier versions
\def\PYZat{@}
\def\PYZlb{[}
\def\PYZrb{]}
\makeatother


    % For linebreaks inside Verbatim environment from package fancyvrb.
    \makeatletter
        \newbox\Wrappedcontinuationbox
        \newbox\Wrappedvisiblespacebox
        \newcommand*\Wrappedvisiblespace {\textcolor{red}{\textvisiblespace}}
        \newcommand*\Wrappedcontinuationsymbol {\textcolor{red}{\llap{\tiny$\m@th\hookrightarrow$}}}
        \newcommand*\Wrappedcontinuationindent {3ex }
        \newcommand*\Wrappedafterbreak {\kern\Wrappedcontinuationindent\copy\Wrappedcontinuationbox}
        % Take advantage of the already applied Pygments mark-up to insert
        % potential linebreaks for TeX processing.
        %        {, <, #, %, $, ' and ": go to next line.
        %        _, }, ^, &, >, - and ~: stay at end of broken line.
        % Use of \textquotesingle for straight quote.
        \newcommand*\Wrappedbreaksatspecials {%
            \def\PYGZus{\discretionary{\char`\_}{\Wrappedafterbreak}{\char`\_}}%
            \def\PYGZob{\discretionary{}{\Wrappedafterbreak\char`\{}{\char`\{}}%
            \def\PYGZcb{\discretionary{\char`\}}{\Wrappedafterbreak}{\char`\}}}%
            \def\PYGZca{\discretionary{\char`\^}{\Wrappedafterbreak}{\char`\^}}%
            \def\PYGZam{\discretionary{\char`\&}{\Wrappedafterbreak}{\char`\&}}%
            \def\PYGZlt{\discretionary{}{\Wrappedafterbreak\char`\<}{\char`\<}}%
            \def\PYGZgt{\discretionary{\char`\>}{\Wrappedafterbreak}{\char`\>}}%
            \def\PYGZsh{\discretionary{}{\Wrappedafterbreak\char`\#}{\char`\#}}%
            \def\PYGZpc{\discretionary{}{\Wrappedafterbreak\char`\%}{\char`\%}}%
            \def\PYGZdl{\discretionary{}{\Wrappedafterbreak\char`\$}{\char`\$}}%
            \def\PYGZhy{\discretionary{\char`\-}{\Wrappedafterbreak}{\char`\-}}%
            \def\PYGZsq{\discretionary{}{\Wrappedafterbreak\textquotesingle}{\textquotesingle}}%
            \def\PYGZdq{\discretionary{}{\Wrappedafterbreak\char`\"}{\char`\"}}%
            \def\PYGZti{\discretionary{\char`\~}{\Wrappedafterbreak}{\char`\~}}%
        }
        % Some characters . , ; ? ! / are not pygmentized.
        % This macro makes them "active" and they will insert potential linebreaks
        \newcommand*\Wrappedbreaksatpunct {%
            \lccode`\~`\.\lowercase{\def~}{\discretionary{\hbox{\char`\.}}{\Wrappedafterbreak}{\hbox{\char`\.}}}%
            \lccode`\~`\,\lowercase{\def~}{\discretionary{\hbox{\char`\,}}{\Wrappedafterbreak}{\hbox{\char`\,}}}%
            \lccode`\~`\;\lowercase{\def~}{\discretionary{\hbox{\char`\;}}{\Wrappedafterbreak}{\hbox{\char`\;}}}%
            \lccode`\~`\:\lowercase{\def~}{\discretionary{\hbox{\char`\:}}{\Wrappedafterbreak}{\hbox{\char`\:}}}%
            \lccode`\~`\?\lowercase{\def~}{\discretionary{\hbox{\char`\?}}{\Wrappedafterbreak}{\hbox{\char`\?}}}%
            \lccode`\~`\!\lowercase{\def~}{\discretionary{\hbox{\char`\!}}{\Wrappedafterbreak}{\hbox{\char`\!}}}%
            \lccode`\~`\/\lowercase{\def~}{\discretionary{\hbox{\char`\/}}{\Wrappedafterbreak}{\hbox{\char`\/}}}%
            \catcode`\.\active
            \catcode`\,\active
            \catcode`\;\active
            \catcode`\:\active
            \catcode`\?\active
            \catcode`\!\active
            \catcode`\/\active
            \lccode`\~`\~
        }
    \makeatother

    \let\OriginalVerbatim=\Verbatim
    \makeatletter
    \renewcommand{\Verbatim}[1][1]{%
        %\parskip\z@skip
        \sbox\Wrappedcontinuationbox {\Wrappedcontinuationsymbol}%
        \sbox\Wrappedvisiblespacebox {\FV@SetupFont\Wrappedvisiblespace}%
        \def\FancyVerbFormatLine ##1{\hsize\linewidth
            \vtop{\raggedright\hyphenpenalty\z@\exhyphenpenalty\z@
                \doublehyphendemerits\z@\finalhyphendemerits\z@
                \strut ##1\strut}%
        }%
        % If the linebreak is at a space, the latter will be displayed as visible
        % space at end of first line, and a continuation symbol starts next line.
        % Stretch/shrink are however usually zero for typewriter font.
        \def\FV@Space {%
            \nobreak\hskip\z@ plus\fontdimen3\font minus\fontdimen4\font
            \discretionary{\copy\Wrappedvisiblespacebox}{\Wrappedafterbreak}
            {\kern\fontdimen2\font}%
        }%

        % Allow breaks at special characters using \PYG... macros.
        \Wrappedbreaksatspecials
        % Breaks at punctuation characters . , ; ? ! and / need catcode=\active
        \OriginalVerbatim[#1,codes*=\Wrappedbreaksatpunct]%
    }
    \makeatother

    % Exact colors from NB
    \definecolor{incolor}{HTML}{303F9F}
    \definecolor{outcolor}{HTML}{D84315}
    \definecolor{cellborder}{HTML}{CFCFCF}
    \definecolor{cellbackground}{HTML}{F7F7F7}

    % prompt
    \makeatletter
    \newcommand{\boxspacing}{\kern\kvtcb@left@rule\kern\kvtcb@boxsep}
    \makeatother
    \newcommand{\prompt}[4]{
        {\ttfamily\llap{{\color{#2}[#3]:\hspace{3pt}#4}}\vspace{-\baselineskip}}
    }
    

    
    % Prevent overflowing lines due to hard-to-break entities
    \sloppy
    % Setup hyperref package
    \hypersetup{
      breaklinks=true,  % so long urls are correctly broken across lines
      colorlinks=true,
      urlcolor=urlcolor,
      linkcolor=linkcolor,
      citecolor=citecolor,
      }
    % Slightly bigger margins than the latex defaults
    
    \geometry{verbose,tmargin=1in,bmargin=1in,lmargin=1in,rmargin=1in}
    
    

\begin{document}
    
    \maketitle
    
    

    
    \hypertarget{laboratory-1---fluid-properties}{%
\section{Laboratory 1 - Fluid
Properties}\label{laboratory-1---fluid-properties}}

:::\{admonition\} Course Website
\url{http://54.243.252.9/ce-3105-webroot/} :::

    \hypertarget{readings}{%
\subsection{Readings}\label{readings}}

\begin{enumerate}
\def\labelenumi{\arabic{enumi}.}
\tightlist
\item
  \href{http://54.243.252.9/ce-3105-webroot/2-Exercises/laboratory1/EFM-2.pdf}{Engineering
  Fluid Mechanics - Chapter 2}
\item
  \href{http://54.243.252.9/ce-3105-webroot/2-Exercises/laboratory1/EFM-22.pdf}{Engineering
  Fluid Mechanics - Chapter 11 (Cd for spheres, p.414)}
\item
  \href{https://mech.at.ua/HolmanICS.pdf}{Holman, J.P. (2012)
  \emph{Experimental Methods for Engineers}, 8th Ed. (Chapters 1-3)}
\end{enumerate}

    \hypertarget{videos}{%
\subsection{Videos}\label{videos}}

\begin{enumerate}
\def\labelenumi{\arabic{enumi}.}
\tightlist
\item
  \href{https://www.youtube.com/watch?v=WwV-azCJWis}{Laboratory 1
  Instructional Video by Dr.~Uddameri}
\item
  \href{https://www.youtube.com/watch?v=NJhxwlVKong}{Measuring Liquid
  Density (YouTube)}
\item
  \href{https://www.youtube.com/shorts/DsYuumidARg}{Measuring Viscosity
  Falling Sphere (YouTube)}
\end{enumerate}

    \hypertarget{outline}{%
\subsection{Outline}\label{outline}}

    

    \hypertarget{background}{%
\subsection{Background}\label{background}}

A fluid has certain characteristics by which its physical condition may
be described. These characteristics are called properties of the fluid.

Properties help describe the ``state'' of the system under study. A
system is whatever is being studied or analyzed, anything not part of
the system is part of the surroundings. The boundary is the imagined
surface that separates the system from its surroundings.

Systems are described by specifying values that characterize the system;
these values are called properties. A \textbf{property} is a measurable
characteristic of a system that depends only on the present conditions
within the system (called the \textbf{state} of the system). The state
of a system means the conditions of the system as defined by specifying
its properties.

Properties can be organized into categories, one of which is
\textbf{material properties}; the properties explored in this laboratory
are selected material properties of some liquids

    \hypertarget{mass-density}{%
\subsection{Mass Density}\label{mass-density}}

Mass density, or just plain density, denoted by the symbol \(\rho\), is
a fundamental property of all substances, including fluids.

It is defined as the mass per unit volume of a material and can be
expressed as an equation in words:

\[\text{Density} = \frac{\text{Mass of the Fluid}}{\text{Volume occupied by the fluid}}\]

Or using Greek letters, mathematically as:

\[\rho = \frac{M}{V}\]

Units: \(\mathrm{kg/m^3}\) (SI), \(\mathrm{lbm/ft^3}\) (FPS).

In the context of fluids, density provides critical insight into how
fluids behave under various conditions, such as flow through pipes,
buoyancy, and pressure variations with depth.

For instance, the density of water at standard temperature and pressure
(STP) is approximately \(1000~kg/m^3\), while air has a much lower
density of about \(1.2~kg/m^3\).

\hypertarget{measuring-density-of-fluids}{%
\subsubsection{Measuring Density of
Fluids}\label{measuring-density-of-fluids}}

Practical methods to measure the density of a liquid include:

\textbf{Gravimetric Method}

This method involves directly measuring the mass and volume of a liquid
sample and using the formula \(\rho = \frac{M}{V}\).

Steps: - Weigh an empty container on a precision scale and record its
mass. - Fill the container with the fluid and weigh it again to
determine the mass of the fluid (\(M\)). - Measure the volume (\(V\)) of
the fluid using a graduated cylinder or volumetric flask. - Calculate
the density.

\emph{Example:}

If \(M=0.25 kg\) and \(V=0.00025 m^3\), the density is:

\(\rho=\frac{0.250~kg}{0.00025~m^3}=1000 kg/m3\)

\textbf{Hydrometer Method}

A hydrometer is a floating instrument calibrated to measure fluid
density directly.

Steps: - Submerge the hydrometer in the fluid. - Read the density from
the scale where the hydrometer's surface level aligns with the fluid.

This method is especially useful for liquids like alcohol, brines, and
oils.

\textbf{Pycnometer Method}

A pycnometer is a specialized container with a known volume used for
highly precise density measurements.

Steps: - Weigh the empty, dry pycnometer. - Fill it with the fluid and
weigh it again to determine the mass of the fluid. - Calculate the
density using the known volume of the pycnometer.

The pycnometer is clearly a gravimetric technique, but eliminates the
need to measure the volume, as the device itself performs this
measurement directly

\hypertarget{importance-of-density-in-fluid-mechanics}{%
\subsubsection{Importance of Density in Fluid
Mechanics}\label{importance-of-density-in-fluid-mechanics}}

Understanding density is critical for:

\begin{itemize}
\tightlist
\item
  Buoyancy Analysis: The density difference between a fluid and a
  submerged object determines whether it floats or sinks.
\item
  Hydrostatics: Density directly influences the pressure variation in a
  fluid column, P=ρghP=ρgh.
\item
  Flow Dynamics: Density is a key parameter in determining the Reynolds
  number, which predicts flow behavior (laminar or turbulent).
\end{itemize}

:::\{admonition\} Temperature Matters At a given temperature and
pressure the density of a given liquid is constant. Let us say we keep
pouring some liquid into a beaker, as the mass increases so does the
volume whereas density, which is the ratio of mass to volume, stays
constant.

As such all experimental determinations of density require that the
temperature of the liquid be measured. :::

Accurately measuring and understanding fluid density, informs engineers
so they can design systems involving fluid transport, storage, and
control.

    \hypertarget{specific-weight}{%
\subsection{Specific Weight}\label{specific-weight}}

\textbf{Specific Weight} is the \textbf{weight} per unit volume of the
material. Remember that weight is a force obtained by multiplying mass
and gravitational acceleration (g).

As an equation in words:

\[\text{Specific Weight} = \frac{\text{weight}}{\text{volume}}\]

or more conventionally

\[\gamma = \frac{W}{V} = \frac{m \cdot g}{V}\]

At a given temperature, pressure and location, the specific weight of a
fluid is constant. However, the acceleration due to gravity varies
slightly with location. The specific weight of a fluid is slightly lower
at the poles than at the equator even when the temperature and pressure
of the fluid are the same at both locations.

    \hypertarget{specific-gravity}{%
\subsection{Specific Gravity}\label{specific-gravity}}

\textbf{Specific Gravity} is another important fluid property that is
defined as the ratio of the density of a fluid to the density of water
at the same temperature.

Clearly, the specific gravity is equal to 1.0 for water. Fluids denser
than water have a specific gravity greater than 1 while those lighter
than water have specific gravity less than 1.

\[SG = \frac{\rho_s}{\rho_{H2O}} \]

Being a ratio of two densities, specific gravity is a dimensionless
quantity. Specific gravity can tell us whether an object will float or
sink in water. Specific Gravity also provides consistency to compare
fluids across different units.

    \hypertarget{viscosity}{%
\subsection{Viscosity}\label{viscosity}}

Viscosity is a measure of a fluid's resistance to deformation or flow
under an applied shear force. It quantifies the internal friction
between adjacent layers of fluid that are moving at different
velocities. Viscosity plays a central role in fluid mechanics as it
affects flow characteristics, energy losses, and the behavior of fluids
in engineering systems.

There are two primary types of viscosity:

\hypertarget{dynamic-viscosity-mu}{%
\subsubsection{\texorpdfstring{Dynamic Viscosity
(\(\mu\))}{Dynamic Viscosity (\textbackslash mu)}}\label{dynamic-viscosity-mu}}

Dynamic viscosity measures the tangential force per unit area (ττ)
required to move one fluid layer relative to another at a unit velocity
while maintaining a unit distance separation. It follows Newton's law of
viscosity:

\(\tau = \mu \frac{\partial y}{\partial u}\)

Where:

\begin{itemize}
\tightlist
\item
  \(\tau\): is the applied shear stress (\(Pa\))
\item
  \(\mu\): is the dynamic viscosity (\(Pa \cdot s\))
\item
  \(\frac{\partial y}{\partial u}\): is the velocity gradient
  perpendicular to the flow.
\end{itemize}

\hypertarget{kinematic-viscosity-nu}{%
\subsubsection{\texorpdfstring{Kinematic Viscosity
(\(\nu\))}{Kinematic Viscosity (\textbackslash nu)}}\label{kinematic-viscosity-nu}}

Kinematic viscosity is the ratio of dynamic viscosity to fluid density:

\(\nu = \frac{\mu}{\rho}\)

Units: \(m^2/s\) in SI.

\hypertarget{measuring-viscosity}{%
\subsubsection{Measuring Viscosity}\label{measuring-viscosity}}

Several methods can be used to measure viscosity. In this laboratory, we
will use on Stoke's law due to its feasibility, but additional methods
are included for context.

\textbf{Stoke's Law (Falling Sphere Method)}

Principle: A small sphere is dropped into a fluid, and its terminal
velocity (\(u\)) is recorded. Under terminal velocity conditions, the
drag force equals the sum of buoyant and gravitational forces. From
Stoke's law, the dynamic viscosity is calculated as:

\(\mu=\frac{2r^2(\rho_s−\rho_f)g}{9u}\)

Where: - \(r\): Sphere radius (mm) - \(\rho_s,\rho_f\): Densities of the
sphere and fluid (\(kg/m^3\)) - \(g\): Gravitational acceleration
(\(m/s^2\)) - \(u\): Terminal velocity (\(m/s\))

Setup: Use a tall column of fluid, small spheres, and a stopwatch to
measure the sphere's falling time over a known distance.

\textbf{Capillary Tube Viscometer}

Principle: Measures the time (\(t\)) it takes for a fluid to flow
through a capillary under gravity or applied pressure. Viscosity is
calculated as:

\(\mu=K \cdot t\)

Where \(K\) is a calibration constant based on the tube's geometry and
fluid properties.

Use: Common for low-viscosity fluids like water and oils.

\textbf{Rotational Viscometer}

\begin{itemize}
\tightlist
\item
  Principle: Measures the torque required to rotate an object (e.g.,
  spindle or cylinder) in a fluid at a constant speed. The torque is
  proportional to the fluid's viscosity.
\item
  Use: Suitable for measuring a wide range of viscosities, including
  non-Newtonian fluids.
\end{itemize}

\textbf{Inclined Plane Method}

\begin{itemize}
\tightlist
\item
  Principle: A known volume of liquid is released on an inclined plane,
  and the time it takes to travel a set distance is recorded. The
  viscosity is inferred from the relationship between time, fluid
  properties, and inclination angle.
\item
  Use: While less precise, this method is a feasible alternative for
  determining relative viscosities in a laboratory.
\end{itemize}

\hypertarget{importance-of-viscosity-in-fluid-mechanics}{%
\subsubsection{Importance of Viscosity in Fluid
Mechanics}\label{importance-of-viscosity-in-fluid-mechanics}}

Viscosity influences several critical aspects of fluid behavior,
including:

\begin{itemize}
\tightlist
\item
  Energy Losses: Higher viscosity increases resistance to flow, leading
  to greater energy dissipation.
\item
  Laminar and Turbulent Flow: The Reynolds number, which predicts flow
  regimes, depends on viscosity.
\item
  Practical Applications: From lubrication in machinery to fluid
  transport in pipelines, viscosity is a key parameter.
\end{itemize}

:::\{admonition\} Temperature Matters Viscosity is strongly
temperature-dependent. Consider how honey flows more easily when warmed
compared to when it's at room temperature or chilled. Similarly, engine
oil thickens in cold weather, making it harder for car engines to start.

As such, all experimental determinations of viscosity require that the
temperature of the liquid be measured. :::

By accurately measuring viscosity, engineers can design efficient
systems that minimize losses and optimize performance. In this lab, we
will use Stoke's law to measure the viscosity of different fluids, but
the insights extend to various industrial and natural applications.

    \hypertarget{viscosity}{%
\subsection{Viscosity}\label{viscosity}}

\textbf{Viscosity} quantifies the ability of the fluid to resist shear
stress (i.e., internal resistance). One can also conceptualize viscosity
as the frictional forces that exist between two layers of fluid that are
in relative motion.

\textbf{Dynamic Viscosity} measures the tangential force per unit area
required to move one horizonal plane relative to another at a unit
velocity when maintaining unit distance separation. The shear stress
applied causes the fluid to flow (or flow causes stress).

Newton's law of viscosity states that the shear stress, \(\tau\), is
proportional to the velocity gradient (across the flow flow),
\(\frac{du}{dy}\) (see Figure 1). Dynamic viscosity ,\(\mu\), is the
constant of proportionality.

Figure 1: Shear stress conceptual diagram

Newton's law expressed as an equation is:
\[\tau = \mu \cdot \frac{du}{dy} \]

thus dynamic viscosity is the ratio of shear force to the velocity
gradient. It has units of \$ Pa \cdot s = \frac{kg \cdot m}{s^2}
\cdot \frac{s}{m^2}\$.

In cgs system the units of dynamic viscosity is Poise (or more commonly
centipoise, cP).

In US Customary units we express viscosity as
\(\frac{lbf}{ft \cdot s}\).

In practical fluid mechanics, we often encounter the ratio of dynamic
viscosity to density. This term is the \textbf{kinematic viscosity}.

Expressed as an equation in commonly used notation:

\[ \nu = \frac{\mu}{\rho}\]

The kinematic viscosity has SI units of \[\frac{m^2}{s}\].

A useful method to determine viscosity of liquids is to record the rate
at which a sphere will fall through a liquid of interest. Under
equilibrium conditions, the frictional forces experienced by the sphere
will be equal to its weight. The sphere will fall at a constant speed
known as the terminal velocity. The phenomenon is called Stokes law (or
Stokes flow).

A simple force balance is depicted in Figure 2, where the bouyant force
and drag force are equal to the weight of the sphere.

Figure 2: Force balance on a sphere falling at constant velocity

Stokes flow occurs at pretty low Reynolds numbers so the laminar
correlation for the drag coefficient is appropriate

\[ c_d = \frac{24}{Re}+\frac{4}{\sqrt{Re}}+0.4 \]

If the Reynolds number is less than \(\frac{1}{2}\) the drag coefficient
is \(c_d = \frac{24}{Re}\), using this representation of drag the force
balance for the sphere allows us to solve for velocity, \(u\),

\[ u = \frac{g \cdot d^2}{18 \nu}(\sigma-\rho)\]

where, \emph{g} is the acceleration due to gravity, \emph{d} is the
diameter of the sphere, \(\nu\) is the kinematic viscosity, \(\sigma\)
is the density of the sphere, \(\rho\) is the density of the fluid.

We can apply the formula to get an idea of how fast to expect a sphere
to fall if Stokes flow holds. In the experiment we will use Glycerine as
the liquid phase, and small steel spheres the largest is about 2.5
millimeters

    \begin{tcolorbox}[breakable, size=fbox, boxrule=1pt, pad at break*=1mm,colback=cellbackground, colframe=cellborder]
\prompt{In}{incolor}{25}{\boxspacing}
\begin{Verbatim}[commandchars=\\\{\}]
\PY{c+c1}{\PYZsh{} Estimate Sphere Falling Speed assuming laminar flow}
\PY{n}{gravity} \PY{o}{=} \PY{l+m+mf}{9.81} \PY{c+c1}{\PYZsh{}m/s\PYZca{}2}
\PY{n}{viscosity} \PY{o}{=} \PY{l+m+mf}{15.103} \PY{c+c1}{\PYZsh{} Ns/m\PYZca{}2}
\PY{n}{density\PYZus{}liquid} \PY{o}{=} \PY{p}{(}\PY{l+m+mf}{69.5}\PY{o}{/}\PY{l+m+mi}{50}\PY{p}{)}\PY{o}{*}\PY{l+m+mi}{1000}\PY{o}{*}\PY{l+m+mi}{1000} \PY{c+c1}{\PYZsh{}kg/m\PYZca{}3}
\PY{n}{density\PYZus{}sphere} \PY{o}{=} \PY{p}{(}\PY{l+m+mf}{11.350}\PY{p}{)}\PY{o}{*}\PY{l+m+mi}{1000}\PY{o}{*}\PY{l+m+mi}{1000} \PY{c+c1}{\PYZsh{}kg/m\PYZca{}3}
\PY{n}{diameter} \PY{o}{=} \PY{l+m+mf}{0.0125} \PY{c+c1}{\PYZsh{}meters \PYZhy{} nominal 2.5mm}
\PY{n}{upper\PYZus{}support\PYZus{}terminal\PYZus{}speed} \PY{o}{=} \PY{p}{(}\PY{n}{gravity}\PY{o}{*}\PY{n}{diameter}\PY{o}{*}\PY{o}{*}\PY{l+m+mi}{2}\PY{p}{)}\PY{o}{*}\PY{p}{(}\PY{n}{density\PYZus{}sphere}\PY{o}{\PYZhy{}}\PY{n}{density\PYZus{}liquid}\PY{p}{)}\PY{o}{/}\PY{p}{(}\PY{l+m+mf}{18.0}\PY{o}{*}\PY{n}{viscosity}\PY{p}{)}
\PY{n+nb}{print}\PY{p}{(}\PY{l+s+s2}{\PYZdq{}}\PY{l+s+s2}{Stokes flow speed limit = }\PY{l+s+s2}{\PYZdq{}}\PY{p}{,}\PY{n+nb}{round}\PY{p}{(}\PY{n}{upper\PYZus{}support\PYZus{}terminal\PYZus{}speed}\PY{p}{,}\PY{l+m+mi}{6}\PY{p}{)}\PY{p}{,}\PY{l+s+s2}{\PYZdq{}}\PY{l+s+s2}{ millimeters per second}\PY{l+s+s2}{\PYZdq{}}\PY{p}{)}
\end{Verbatim}
\end{tcolorbox}

    \begin{Verbatim}[commandchars=\\\{\}]
Stokes flow speed limit =  56.158131  millimeters per second
    \end{Verbatim}

    So using the above script we conclude that we should be able to make
measurements for spheres as large as 25 mm, using a stopwatch and visual
observation, our spheres are quite a bit smaller, so we should have no
issues.

    

    \hypertarget{laboratory-objectives}{%
\subsection{Laboratory Objectives}\label{laboratory-objectives}}

\begin{enumerate}
\def\labelenumi{\arabic{enumi}.}
\tightlist
\item
  Measure density, specific gravity, and viscosity of various liquids.
\item
  Develop an experimental protocol (step-by-step instructions) to
  measure density, specific gravity, and viscosity of three different
  liquids.
\item
  Upon approval of the protocol, conduct a set of experiments in
  triplicate to measure the density, specific gravity, and viscosity of
  three different liquids.
\item
  Document the experiment(s) into a laboratory report and address the
  following in the report:
\end{enumerate}

\begin{itemize}
\tightlist
\item
  Derive the fall velocity equation, starting from the force balance on
  the sphere and assuming that \(C_D=\frac{24}{Re_D}\)
\item
  Compare your results with tabulated values for density and viscosity
  for the three fluids.
\item
  Experiments are conducted in triplicate, so you can compute mean
  values and standard deviations; what does this information tell us
  about the accuracy of the measurements?, What does it tell us about
  the repeatability of the measurements?
\item
  What are some potential sources of errors in your lab experiments.
  Discuss in the context of measuring density, specific gravity and
  viscosity.
\end{itemize}

The protocol is evaluated to ensure that the envisioned procedures can
be safely conducted under appropriate supervision. The experiments may
produce incomplete results if steps are ommitted.

    \hypertarget{deliverables}{%
\subsection{Deliverables}\label{deliverables}}

\begin{enumerate}
\def\labelenumi{\arabic{enumi}.}
\tightlist
\item
  Develop an \textbf{experimental protocol} (step-by-step instructions)
  to measure density, specific gravity, and viscosity of three different
  liquids. Submitted in advance for instructor approval - the protocol
  will become part of your laboratory report.
\item
  \textbf{Laboratory Report} documenting the actual experiments, and
  other required content including comparison to tabulated values.
\end{enumerate}

    \hypertarget{link-to-laboratory-document}{%
\subsection{Link to Laboratory
Document}\label{link-to-laboratory-document}}

\begin{enumerate}
\def\labelenumi{\arabic{enumi}.}
\tightlist
\item
  \href{http://54.243.252.9/ce-3105-webroot/2-Exercises/laboratory1/laboratory1.html}{Laboratory
  1 Tasklist}
\item
\end{enumerate}

    

    

    \begin{tcolorbox}[breakable, size=fbox, boxrule=1pt, pad at break*=1mm,colback=cellbackground, colframe=cellborder]
\prompt{In}{incolor}{ }{\boxspacing}
\begin{Verbatim}[commandchars=\\\{\}]

\end{Verbatim}
\end{tcolorbox}


    % Add a bibliography block to the postdoc
    
    
    
\end{document}
